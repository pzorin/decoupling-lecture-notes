\documentclass[biblatex]{pzorin-note}
\usepackage{booktabs}
\usepackage{amsmath}
\usepackage{amssymb}
\usepackage{amsthm}
\usepackage{xspace}
\usepackage{mathtools}
\usepackage{etoolbox}
\usepackage{todonotes}

% Companion to \to. Example:
%  Let $ f \from A \to B $ be a function.
\newcommand*{\from}{\colon}


\newcommand*{\Z}{\mathbb{Z}}
\newcommand*{\Q}{\mathbb{Q}}
\def\C{\mathbb{C}}
\newcommand*{\N}{\mathbb{N}}
\newcommand*{\R}{\mathbb{R}}
\newcommand*{\K}{\mathbb{K}}
\newcommand*{\E}{\mathbb{E}}
\newcommand*{\T}{\mathbb{T}}
\newcommand*{\Rp}{\mathbb{R}_+}
\newcommand*{\boundary}{\partial}
\newcommand*{\id}{\mathrm{id}}
\newcommand*{\Id}{\mathrm{Id}}
\newcommand*{\Quot}{\mathrm{Quot}}
\newcommand*{\Gal}{\mathrm{Gal}}
\newcommand*{\hol}[1]{\mathcal{H}(#1)}
\newcommand*{\Rang}[1]{\mathrm{Rang}\left(#1\right)}
\newcommand*{\Mat}{\mathrm{Mat}}
\newcommand*{\End}{\mathrm{End}}
\newcommand*{\Aut}{\mathrm{Aut}}
\newcommand*{\const}{\mathrm{const}}
\newcommand*{\z}{\bar z}
\newcommand*{\inv}{^{-1}}
\newcommand*{\cconv}{\overline{\mathrm{conv}}\,}
\newcommand*{\Union}{\bigcup\limits}
\newcommand*{\Intersection}{\bigcap\limits}
\newcommand*{\union}{\cup}
\newcommand*{\intersection}{\cap}
\newcommand*{\Sum}{\sum\limits}
\newcommand*{\Product}[1]{\prod\limits_{#1}}
\newcommand*{\Prod}{\prod\limits}
\newcommand*{\Meet}{\bigwedge\limits}
\newcommand*{\JOIN}{\bigvee\limits}
\newcommand*{\join}{\vee}
\newcommand*{\meet}{\wedge}
\newcommand*{\with}{\,:\,}
\newcommand*{\unit}[1]{\,\mathrm{#1}}
\newcommand*{\BMO}{\mathrm{BMO}}
\newcommand*{\VMO}{\mathrm{VMO}}
\newcommand*{\Schwartz}{\mathcal{S}}
\newcommand{\one}{\mathbf{1}}
\newcommand*{\widevec}[1]{\overrightarrow{#1}}

% Differential d's
\newcommand{\dif}{\mathop{}\!\mathrm{d}} % \mathop produces two thin spaces, \! removes the trailing one
\newcommand*{\tdif}[3][]{\frac{\dif^{ #1} #2}{\dif { #3}^{ #1}}}
\newcommand*{\ttdif}[2]{\frac{\dif^2 #1}{\dif {#2} ^2}}
\newcommand*{\pdif}[3][]{\frac{\partial^{ #1} #2}{\partial { #3}^{ #1}}}
\newcommand*{\ppdif}[2]{\frac{\partial^{2} #1}{\partial {#2}^{2}}}

\let\sphi\phi % Stroked phi
\renewcommand*{\phi}{\varphi}
\renewcommand*{\epsilon}{\varepsilon}

\def\<{\left\langle}
\def\>{\right\rangle}

% DeclareMathOperator without the repetition
\newcommand*{\DMO}[1]{\expandafter\DeclareMathOperator\csname #1\endcsname {#1}}

\DMO{card}
\DMO{Ran}
\DMO{diam}
\DMO{dist}
\DMO{dom}
\DMO{Ad}
\DMO{ad}
\DMO{Shift}
\DMO{esssup}
\DMO{tr}
\DMO{grad}
\DMO{Kern}
\DMO{Bild}
\DMO{Lin}
\DMO{Hom}
\DMO{lin}
\DMO{sg}
\DMO{sign}
\DMO{ord}
\DMO{supp}
\DMO{ggT}
\DMO{kgV}
\DMO{Res}
\DMO{Ann}
\DMO{Ass}
\DMO{Rad}
\DMO{conv}
\DMO{im}
\DMO{lcm}
\DMO{rank}

% Theorem-like environments
\theoremstyle{plain}
\ifdefined\theorem\else % Already defined by beamer
\newtheorem{theorem}{Theorem}
\newtheorem{thm}[theorem]{Theorem}
\newtheorem{proposition}[theorem]{Proposition}
\newtheorem{prop}[theorem]{Proposition}
\newtheorem{lemma}[theorem]{Lemma}
\newtheorem{lem}[theorem]{Lemma}
\newtheorem{corollary}[theorem]{Corollary}
\newtheorem{cor}[theorem]{Corollary}
\newtheorem{conjecture}[theorem]{Conjecture}
\ifdefined\section % May be not defined in scrlttr2
\numberwithin{equation}{section}
\numberwithin{theorem}{section}
\else\fi
\fi

\theoremstyle{remark}
\ifdefined\remark\else % Already defined by beamer
\newtheorem*{remark}{Remark}
\newtheorem*{claim}{Claim}
\fi
\newtheorem*{speculation}{Speculation}
\newtheorem*{prob}{Problem}
\ifdefined\example\else % Already defined by beamer
\newtheorem*{example}{Example}
\fi
\ifdefined\question\else % Already defined by beamer
\newtheorem*{question}{Question}
\fi

\theoremstyle{definition}
\ifdefined\definition\else % Already defined by beamer
\newtheorem{definition}[theorem]{Definition}
\fi
\newtheorem{defi}[theorem]{Definition}

% Proper names with accents
\newcommand*{\Frechet}{Fr\'echet\xspace} % Fréchet
\newcommand*{\Konig}{K\H{o}nig\xspace} % Kőnig
\newcommand*{\Cesaro}{Ces\`aro\xspace} % Cesàro
\newcommand*{\Folner}{F\o{}lner\xspace} % Følner

% Absolute values and norms using mathtools. \[lr][vV]ert produces correct spacing as opposed to | and \|.
\DeclarePairedDelimiter\abs{\lvert}{\rvert}
\DeclarePairedDelimiter\norm{\lVert}{\rVert}
\DeclarePairedDelimiter\floor{\lfloor}{\rfloor}
\DeclarePairedDelimiter\ceil{\lceil}{\rceil}
\DeclarePairedDelimiterX\innerp[2]{\langle}{\rangle}{#1,#2}
% just to make sure it exists
\providecommand\given{}
% can be useful to refer to this outside \Set
\newcommand\SetSymbol[1][]{%
\nonscript\:#1\vert
\allowbreak
\nonscript\:
\mathopen{}}
\DeclarePairedDelimiterX\Set[1]\{\}{%
\renewcommand\given{\SetSymbol[\delimsize]}
#1
}

% Alexander Perlis | A complement to \smash, \llap, and \rlap
% math.arizona.edu/~aprl/publications/mathclap/
% For comparison, the existing overlap macros:
% \def\llap#1{\hbox to 0pt{\hss#1}}
% \def\rlap#1{\hbox to 0pt{#1\hss}}
\def\clap#1{\hbox to 0pt{\hss#1\hss}}
\def\mathllap{\mathpalette\mathllapinternal}
\def\mathrlap{\mathpalette\mathrlapinternal}
\def\mathclap{\mathpalette\mathclapinternal}
\def\mathllapinternal#1#2{%
\llap{$\mathsurround=0pt#1{#2}$}}
\def\mathrlapinternal#1#2{%
\rlap{$\mathsurround=0pt#1{#2}$}}
\def\mathclapinternal#1#2{%
\clap{$\mathsurround=0pt#1{#2}$}}

\def\PZdefchar#1{
  \expandafter\def\csname frak#1\endcsname{\mathfrak{#1}}
  \expandafter\def\csname bb#1\endcsname{\mathbb{#1}}
  \expandafter\def\csname bf#1\endcsname{\mathbf{#1}}
  \expandafter\def\csname scr#1\endcsname{\mathcal{#1}}
  \expandafter\def\csname cal#1\endcsname{\mathcal{#1}}}
\def\PZdefloop#1{\ifx#1\PZdefloop\else\PZdefchar#1\expandafter\PZdefloop\fi}
\PZdefloop abcdefghijklmnopqrstuvwxyzABCDEFGHIJKLMNOPQRSTUVWXYZ\PZdefloop

% Work around beamer's \pause and amsmath's align incompatibility
% https://tex.stackexchange.com/a/75550/30158
\makeatletter
\let\save@measuring@true\measuring@true
\def\measuring@true{%
  \save@measuring@true
  \def\beamer@sortzero##1{\beamer@ifnextcharospec{\beamer@sortzeroread{##1}}{}}%
  \def\beamer@sortzeroread##1<##2>{}%
  \def\beamer@finalnospec{}%
}
\makeatother

\def\PZdefchar#1{
  \expandafter\def\csname frak#1\endcsname{\mathfrak{#1}}
  \expandafter\def\csname bf#1\endcsname{\mathbf{#1}}
  \expandafter\def\csname scr#1\endcsname{\mathcal{#1}}
  \expandafter\def\csname cal#1\endcsname{\mathcal{#1}}}
\def\PZdefloop#1{\ifx#1\PZdefloop\else\PZdefchar#1\expandafter\PZdefloop\fi}
\PZdefloop abcdefghijklmnopqrstuvwxyzABCDEFGHIJKLMNOPQRSTUVWXYZ\PZdefloop

\def\tA{\tilde{A}}
\def\tD{\tilde{D}}
\def\tV{\tilde{V}}
\def\tGamma{\tilde{\Gamma}}
\def\tgamma{\tilde{\gamma}}
\def\C{\mathbb{C}}

\DeclarePairedDelimiterXPP\EE[1]{\E}{\lparen}{\rparen}{}{\renewcommand\given{\SetSymbol[\delimsize]}#1} % Conditional expectation \EE{ f \given A }

\makeatletter
\newcommand\@avsum[2]{%
  {\sbox0{$\m@th#1\sum$}%
   \vphantom{\usebox0}%
   \ooalign{%
     \hidewidth
     \smash{\vrule height\dimexpr\ht0+1pt\relax depth\dimexpr\dp0+1pt\relax}%
     \hidewidth\cr
     $\m@th#1\sum$\cr
   }%
  }%
}
\newcommand{\avsum}{\mathop{\mathpalette\@avsum\relax}\displaylimits}
\newcommand\@avprod[2]{%
  {\sbox0{$\m@th#1\prod$}%
   \vphantom{\usebox0}%
   \ooalign{%
     \hidewidth
     \smash{\vrule height\dimexpr\ht0+1pt\relax depth\dimexpr\dp0+1pt\relax}%
     \hidewidth\cr
     $\m@th#1\prod$\cr
   }%
  }%
}
\newcommand{\avprod}{\mathop{\mathpalette\@avprod\relax}\displaylimits}
\newcommand{\@avL}[2]{%
\ooalign{{$\m@th#1\mbox{--}$}\cr {$\m@th#1 L$}\cr}}
\newcommand{\avL}{\mathpalette\@avL\relax}
\newcommand{\@avell}[2]{%
\ooalign{{$\m@th#1\mbox{--}$}\cr {$\m@th#1 \ell$}\cr}}
\newcommand{\avell}{\mathpalette\@avell\relax}
\newcommand{\@avD}{%
  \ooalign{{$\mathrm{D}$}\cr \hidewidth\raise.2ex\hbox{$\vert$}\hidewidth\cr}}
\newcommand{\avDec}{\@avD\mathrm{ec}}
\makeatother
\newcommand{\Dec}{\mathrm{Dec}}
\newcommand{\MulDec}{\mathrm{MulDec}}
\newcommand{\RvSq}{\mathrm{RvSq}}
\newcommand{\MlRvSq}{\mathrm{MlRvSq}}
\newcommand{\BL}{\mathrm{BL}}
\newcommand{\BLg}{\mathrm{BL}_{\mathbf{g}}}
\newcommand{\ED}{E^{\calD}}
\newcommand{\Part}[2][]{\calP(\ifstrempty{#1}{}{#1,}#2)} % Partition of #1 at scale #2.
\newcommand{\Cov}[2][\R^{d+n}]{\calB(#1,#2)} % Covering of #1 at scale #2.

\usepackage{esint}
\newcommand{\MK}{\mathcal{K}}
\newcommand{\Tubes}{\mathbf{T}}
\def\FT{\mathcal{F}}

\usepackage{tikz}
% This code defines TikZ "parabola through" path operation, taken from
% https://tex.stackexchange.com/a/429938/30158
% In that post it is explained that TikZ builtin "parabola" is wrong.
\makeatletter
\def\pt@get#1#2{
  \tikz@scan@one@point\pgfutil@firstofone#2\relax%
  \csname pgf@x#1\endcsname=\pgf@x%
  \csname pgf@y#1\endcsname=\pgf@y%
}
\tikzset{
  parabola through/.style={
    to path={{[x={(\pgf@xc,\pgf@yc)}, y=\parabola@y, shift=(\tikztostart)]
      -- (0,0) .. controls (1/3,1/3) and (2/3,1/3) .. (1,0) \tikztonodes}--(\tikztotarget)}
  },
  parabola through/.prefix code={
    \pt@get{a}{(\tikztostart)}\pt@get{b}{#1}\pt@get{c}{(\tikztotarget)}%
    \advance\pgf@xb by-\pgf@xa\advance\pgf@yb by-\pgf@ya%
    \advance\pgf@xc by-\pgf@xa\advance\pgf@yc by-\pgf@ya%
    \pgfmathsetmacro\parabola@y{(\pgf@yc-\pgf@xc/\pgf@xb*\pgf@yb)%
      /(\pgf@xb-\pgf@xc)*\pgf@xc}%
  }
}
\makeatother

\numberwithin{equation}{section}
\theoremstyle{definition}
\newtheorem{hypothesis}[equation]{Hypothesis}

\renewcommand*{\det}{\qopname\relax o{det}} % equivalent to \nolimits

%GENERAL
\newcommand{\eit}{e^{i t \Delta}}
\newcommand{\eir}{e^{i r \Delta}}
\newcommand{\bbS}{\mathbb{S}}

\newcommand{\RefStr}{\mathrm{RefStr}}
\newcommand{\MultRefStr}{\mathrm{MultRefStr}}
\newcommand{\FrStr}{\mathrm{FrStr}}

\newcommand{\bB}{\mathfrak{B}} % ball Big
\newcommand{\bM}{\mathfrak{M}} % ball Medium
\newcommand{\bS}{\mathfrak{S}} % ball Smal
\newcommand{\bBp}{\mathfrak{B}'} % ball Big previous scale
\newcommand{\bMp}{\mathfrak{M}'} % ball Medium previous scale
\newcommand{\bSp}{\mathfrak{S}'} % ball Small previous scale
\newcommand{\bO}{\mathfrak{O}} % ball Out
\newcommand{\bI}{\mathfrak{I}} % ball In
\newcommand{\bOp}{\mathfrak{O}'} % ball Out previous scale
\newcommand{\bIp}{\mathfrak{I}'} % ball In previous scale

\begin{document}

In this note we recover the main results of \cite{MR3842310} and \cite{arxiv:1805.02775} without any pigeonholing.
Also, Proposition~\ref{prop:fractal-strichartz} is slightly more refined than \cite[Proposition 3.1]{arxiv:1805.02775} and combines the results for all dimensions $\alpha$ in one estimate.
Some text is copied from the cited articles.


\subsection*{Notation}
We write $A\lesssim B$ if $A\leq CB$ for some absolute constant $C$, $A \sim B$ if $A\lesssim B$ and $B\lesssim A$; $A\ll B$ if $A$ is much less than $B$; $A\lessapprox B$ if $A\leq C_\epsilon R^\epsilon B$ for any $\epsilon>0, R>1$.
Sometimes we also write $A\lesssim B$ if $A\leq C_\epsilon B$ for some constant $C_\epsilon$ depending on $\epsilon$ (when the dependence on $\epsilon$ is unimportant).

By an $r$-ball (cube) we mean a ball (cube) of radius (side length) $r$.
An $r\times\cdots\times r \times L$-tube (box) means a tube (box) with radius (short sides length) $r$ and length $L$.
For a set $\calS$, $\#\calS$ denotes its cardinality.

\section{Wave packet decomposition}
We ignore all tails and pretend that functions are compactly suported in space and in frequency.
Real wave packet decomposition can be found in \cite{MR3454378}.

Let $f \in L^{2}(\R^{n})$ with $\supp \hat{f} \subset B^{n}(0,1)$.
If $\tau \subset B^{n}(0,1)$ is a cube, then $f_{\tau}$ is defined by $\widehat{f_{\tau}} = \hat{f} \one_{\tau}$.
Given $1 \leq K \leq R^{1/2}$, $\tau$ of side length $1/K$, and $B \subset \R^{n}$ ball of radius $R/K$ we define $f_{\tau,B}$ to be $f_{\tau}$ smoothly truncated to $B$, so that its Fourier support remains inside $2\tau$.
Let $\bBp = \bBp_{\tau, B}$ be the tube of length $R$ and radius $R/K$ whose central line starts at $(c(B),0)$ and points in the direction $G(c(\tau))$.
Here $c(\tau)$ is the center of $\tau$ and for $\xi \in B^{n}(0,1)$ we write $G(\xi):=\frac{(-2\xi,1)}{\abs{(-2\xi,1)}} \in \bbS^n$.
There is a one-to-one correspondence between $\bBp$ and $(\tau, B)$, so we will also write $f_{\bBp} = f_{\tau,B}$.
Then $\eit f_{\bBp}$ is essentially supported in $\bBp$ for $t\leq R$.
Also,
\begin{equation}
\label{eq:orthogonality}
\ell^{2}_{\bBp} \norm{f_{\bBp}}_{2}
\lessapprox
\norm{f}_{2}.
\end{equation}


\section{Linear refined Strichartz}
\label{sec:LRS}
In the following table we list the names that we use for regions of different sizes in $\R^{n+1}$.
If two sizes are given, then the first one is the size in the space direction and the second in the time direction.
Tubes (regions that are longer in the time direction than they are wide in the space direction) can also have different orientations.
\begin{center}
\begin{tabular}{ccc} \toprule
  Object & Size before scaling & Size after scaling\\ \midrule
  $\bB$ & $R$ & \\
  Slab & $R \times R^{1/2}$ & \\
  $\bM$ & $R^{1/2}$ & \\
  $\bBp$ & $R^{3/4} \times R$ & $R^{1/2}$\\
  $slab$ & $R^{3/4}$ & $R^{1/2} \times R^{1/4}$\\
  $\bMp$ & $R^{1/2} \times R^{3/4}$ & $R^{1/4}$\\
  \bottomrule
\end{tabular}
\end{center}
The following possible inclusions will be relevant:
\[
\bB \supset Slab \supset \bM, \quad
\bB \supset \bBp \supset slab \supset \bMp \supset \bM.
\]
If we write that a tube is contained in another tube we imply that also the directions match.

\begin{theorem} [Linear refined Strichartz in dimension $n+1$] \label{thm-LRS}
Let $p = \frac{2(n+2)}{n}$.
Suppose that $f: \R^n \rightarrow \C$ has frequency supported in $B^n(0,1)$.
Then
\begin{equation}
\label{LRS}
\ell^{p}_{Slab \subset \bB} \ell^{2}_{\bM \subset Slab} \norm{ \eit f }_{L^{p}(\bM)}
\lessapprox
\norm{ f }_{L^2}.
\end{equation}
\end{theorem}
\cite[Theorem 3.1]{MR3842310} can be recovered using the fact that $-\frac{1}{n+2} = 1/p - 1/2$.
The nice thing about our version is that lots of dyadic pigeonholing is replaced by Minkowski's inequality.

The proof uses the Bourgain-Demeter $\ell^2$ decoupling theorem, together with induction on the radius and parabolic rescaling.
First we recall the decoupling result of Bourgain and Demeter.
\begin{theorem}[{\cite{MR3374964}}]
\label{bourdem}
Let $n\geq 1$ and $p:=\frac{2(n+2)}{n}$.
Suppose that the $R^{-1}$-neighborhood of the unit paraboloid in
$\R^{n+1}$ is divided into $R^{n/2}$ disjoint rectangular boxes $\tau$, each with dimensions $R^{-1/2}\times\dotsm\times R^{-1/2} \times R^{-1}$.
Suppose $\widehat F_\tau$ is supported in $\tau$ and $F = \sum_\tau F_\tau$.
Then on each ball $B \subset \R^{n+1}$ of radius $\geq R$ we have
\[
\norm{ F }_{L^{p}(B)}
\lessapprox
\big( \sum_\tau \norm{ F_\tau }_{L^{p}(w_{B})}^2 \big)^{1/2}.
\]
\end{theorem}

\begin{proof}[Proof of Theorem~\ref{thm-LRS}]
For $R\geq 1$ let $\RefStr(R)$ denote the smallest constant such that the inequality
\begin{equation} \label{LRS-const}
\ell^{p}_{Slab} \ell^{2}_{\bM \subset Slab} \norm{ \eit f }_{L^{p}(\bM)}
\leq
\RefStr(R) \norm{ f }_{L^2}
\end{equation}
holds for that value of $R$.
It is clear that $\RefStr(R) \lesssim R^{O(1)}$.
It suffices to show that
\begin{equation}\label{eq:RS-recursion}
\RefStr(R) \lessapprox \RefStr(R^{1/2}).
\end{equation}
Applying this inequality $\log\log R$ times and using the trivial estimate at the end we will obtain $\RefStr(R) \lessapprox 1$.

We make a wave packet decomposition $f = \sum_{\bBp} f_{\bBp}$ at scale $K=R^{1/4}$.
For each cap $\tau$ and each $\bM$ there is approximately one $\bBp$ with direction $\tau$ that intersects $\bM$.
Hence the decoupling theorem tells us that
\begin{equation}
\norm{ e^{i t \Delta} f }_{L^{p}(\bM)}
\lessapprox
\ell^{2}_{\bBp \supset \bM} \norm{ \eit f_{\bBp} }_{L^{p}(\bM)}.
\end{equation}
Substituting this on the left-hand side of \eqref{LRS} we obtain
\[
LHS\eqref{LRS}
\lessapprox
\ell^{p}_{Slab} \ell^{2}_{\bM \subset Slab} \ell^{2}_{\bBp} \one_{\bM \subset \bBp} \norm{ \eit f_{\bBp} }_{L^{p}(\bM)}.
\]
By Minkowski's inequality this is
\[
\leq
\ell^{2}_{\bBp} \ell^{p}_{Slab} \ell^{2}_{\bM \subset Slab} \one_{\bM \subset \bBp} \norm{ \eit f_{\bBp} }_{L^{p}(\bM)}.
\]
By orthogonality \eqref{eq:orthogonality}, for a fixed $\bBp$ it suffices to show
\[
\ell^{p}_{Slab} \ell^{2}_{\bM \subset Slab} \one_{\bM \subset \bBp} \norm{ \eit f_{\bBp} }_{L^{p}(\bM)}
\lessapprox
\RefStr(R^{1/2}) \norm{f_{\bBp}}_{2}.
\]
Now we partition $\bBp$ into slabs denoted by $slab$ (with a small $s$) and each $slab$ into tubes denoted by $\bMp$.
Then since the long side of $\bBp$ is not very close to being horizontal, each $Slab$ intersects only boundedly many $slab \subset \bBp$, and for every choice of $Slab$ and $\bMp$ there are only boundedly many $\bM \subset Slab \cap \bMp$.
Hence
\begin{multline*}
\ell^{p}_{Slab} \ell^{2}_{\bM \subset Slab} \one_{\bM \subset \bBp} \norm{ \eit f_{\bBp} }_{L^{p}(\bM)}\\
=
\ell^{p}_{Slab} \ell^{2}_{slab} \ell^{2}_{\bM \subset Slab \cap slab} \one_{\bM \subset \bBp} \norm{ \eit f_{\bBp} }_{L^{p}(\bM)}\\
\lesssim
\ell^{p}_{slab} \ell^{p}_{Slab} \ell^{2}_{\bM \subset Slab \cap slab} \one_{\bM \subset \bBp} \norm{ \eit f_{\bBp} }_{L^{p}(\bM)}\\
=
\ell^{p}_{slab} \ell^{p}_{Slab} \ell^{2}_{\bMp \subset slab} \ell^{2}_{\bM \subset Slab \cap \bMp} \one_{\bM \subset \bBp} \norm{ \eit f_{\bBp} }_{L^{p}(\bM)}\\
\leq
\ell^{p}_{slab} \ell^{2}_{\bMp \subset slab} \ell^{p}_{Slab} \ell^{2}_{\bM \subset Slab \cap \bMp} \one_{\bM \subset \bBp} \norm{ \eit f_{\bBp} }_{L^{p}(\bM)}\\
\lesssim
\ell^{p}_{slab} \ell^{2}_{\bMp \subset slab} \ell^{p}_{Slab} \ell^{p}_{\bM \subset Slab \cap \bMp} \one_{\bM \subset \bBp} \norm{ \eit f_{\bBp} }_{L^{p}(\bM)}\\
\lesssim
\ell^{p}_{slab} \ell^{2}_{\bMp \subset slab} \ell^{p}_{\bM \subset \bMp} \one_{\bM \subset \bBp} \norm{ \eit f_{\bBp} }_{L^{p}(\bM)}\\
\lesssim
\ell^{p}_{slab} \ell^{2}_{\bMp \subset slab} \norm{ \eit f_{\bBp} }_{L^{p}(\bMp)}.
\end{multline*}
The right-hand side is analogous to the left-hand side of \eqref{LRS}.
By parabolic rescaling the right-hand side can be estimated by
\[
\RefStr(R^{1/2}) \norm{f_{\bBp}}_{2}.
\]
This closes the induction on radius and completes the proof.
\end{proof}

\section{Mixed norm multilinear Kakeya}
By scaling the inequality
\[
\int \prod_{j=0}^{n} f_{j}(x_{0},\dotsc,x_{n})
\lesssim
\prod_{j=0}^{n} \norm{ \norm{f_{j}}_{L^{p}_{x_{j}}} }_{L^{q}_{x_{(j)}}}
\]
can only hold if $q=np'$.
The usual Loomis--Whitney inequality is the case $p=\infty$.
The case $p=q=n+1$ is just H\"older's inequality.
The intermediate cases follow by interpolation.

We need a similar mixed norm version of the Bennett--Carbery--Tao multilinear Kakeya inequality which we now recall.

\begin{theorem}[{\cite{MR2275834}}] \label{MK}
Let $S_j \subset \bbS^{m-1}$, $j=1,\dotsc,k$, be $\nu$-transverse in the sense that for any vectors $v_j\in S_j$ we have
\[
\abs{v_1\wedge \dotsb \wedge v_k} \geq \nu.
\]
Suppose that $l_{j,a}$ are lines in $\R^m$ and that the direction of $l_{j,a}$ lies in $S_j$.
Let $T_{j,a}$ be the characteristic function of the $1$-neighborhood of $l_{j,a}$.
Let $Q_s$ denote a cube of side length $S \geq 1$.
Then for any $\epsilon>0$ we have
\[
\int_{Q_s} \prod_{j=1}^{k} \big(\sum_{a=1}^{N_j} T_{j,a}\big)^{1/(k-1)} \leq C_\epsilon {\rm Poly}(\nu^{-1}) S^{\epsilon} \prod_{j=1}^{k} N_j^{1/(k-1)}\,.
\]
\end{theorem}

\begin{corollary}\label{cor:mixed-norm-mult-Kakeya}
If in the situation of Theorem~\ref{MK} we have $n \leq r/\gamma \leq n+1$ and $r/\gamma = n(p/\gamma)'$, then for any sequences $c_{j,a,Q}$ with $1$-boxes $Q$ we obtain
\[
\ell^{\gamma}_{Q} \prod_{j=0}^{n} \ell^{r}_{a} \one_{Q \subset T_{j,a}} c_{j,a,Q}
\lessapprox
\prod_{j=0}^{n} \ell^{r}_{a} \ell^{p}_{Q \subset T_{j,a}} c_{j,a,Q}
\]
\end{corollary}
\begin{proof}
Without loss of generality $\gamma=1$.
The case $r=n$, $p=\infty$ is Theorem~\ref{MK}.
The case $r=p=n+1$ is H\"older's inequality.
To obtain the general case we dualize with $\ell^{\infty}_{Q}\ell^{r'}_{a}$ sequences and use complex interpolation.
\end{proof}
The scaling condition can be reformulated as
\begin{equation}
\label{eq:mixed-norm-mult-Kakeya:scaling}
r/\gamma = n(p/\gamma)'
\iff
\frac{n}{r} + \frac{1}{p} = \frac{1}{\gamma}.
\end{equation}

\section{Multilinear refined Strichartz estimate}\label{sec:kRS}

\begin{theorem} [$n+1$-linear refined Strichartz in dimension $n+1$] \label{thm-kRS}
Let $p:=\frac{2(n+2)}{n}$.
Let $f_j: \R^n \rightarrow \C$, $j=0,\dotsc,n$, with $G(\supp \widehat{f_{j}})$ being $\nu$-transverse.
Then
\begin{equation} \label{kRS}
\Bigl( \sum_{\bM \subset \bB} \prod_{j=0}^{n} \norm{ \eit f_{j} }_{L^{p}(\bM)}^{\frac{1}{n+1} \cdot \beta} \Bigr)^{1/\beta}
\lessapprox \nu^{-C}
\prod_{j=0}^{n} \norm{f_{j}}_{2}^{1/(n+1)}
\end{equation}
with $\beta = p \frac{n+1}{n+3} = 2 \frac{(n+2)(n+1)}{(n+3)n}$.
\end{theorem}

\cite[Theorem 4.2]{MR3842310} can be again recovered since $\frac{1}{\beta} = \frac{1}{p} + \frac{n}{(n+1)(n+2)}$.
Here $\beta$ arises from
\[
\frac{n+1}{\beta} = \frac{1}{\gamma} = \frac{n}{2} + \frac{1}{p}.
\]

\begin{proof}[Proof of Theorem~\ref{thm-kRS}]
We begin by writing
\[
LHS\eqref{kRS}^{n+1}
=
\ell^{\gamma}_{\bM} \prod_{j=0}^{n} \norm{ \eit f_{j} }_{L^{p}(\bM)}
\sim
\ell^{\gamma}_{slab} \ell^{\gamma}_{\bM\subset slab} \prod_{j=0}^{n} \norm{ \eit f_{j} }_{L^{p}(\bM)}
\]
with $\gamma = \beta/(n+1) = \frac{p}{n+3} = \frac{2(n+2)}{n(n+3)}$.
Now we temporarily fix $slab$.
Denote by $\bBp_{j}$ the set of $\bBp$ whose direction is in $G(\supp \widehat{f_{j}})$.
By decoupling
\begin{multline*}
\ell^{\gamma}_{\bM\subset slab} \prod_{j=0}^{n} \norm{ \eit f_{j} }_{L^{p}(\bM)}
\lessapprox
\ell^{\gamma}_{\bM \subset slab} \prod_{j=0}^{n} \ell^{2}_{\bBp \in \bBp_{j} : \bM \subset \bBp} \norm{ \eit f_{\bBp} }_{L^{p}(\bM)}\\
\sim
\ell^{\gamma}_{\bM \subset slab} \prod_{j=0}^{n} \ell^{2}_{\bBp \in \bBp_{j}, \bMp \subset \bBp \cap slab : \bM \subset \bMp} \norm{ \eit f_{\bBp} }_{L^{p}(\bM)}
\end{multline*}
Since $n < 2/\gamma = \frac{n(n+3)}{n+2} < n+1$ and \eqref{eq:mixed-norm-mult-Kakeya:scaling} holds, by Corollary~\ref{cor:mixed-norm-mult-Kakeya} with tubes $\bMp$ and boxes $\bM$ this is
\[
\lessapprox
\prod_{j=0}^{n} \ell^{2}_{\bBp \in \bBp_{j}, \bMp \subset \bBp \cap slab} \ell^{p}_{\bM \subset \bMp} \norm{\eit f_{\bBp}}_{L^{p}(\bM)}
=
\prod_{j=0}^{n} \ell^{2}_{\bBp \in \bBp_{j}, \bMp \subset \bBp \cap slab} \norm{\eit f_{\bBp}}_{L^{p}(\bMp)}
\]
and
\[
LHS\eqref{kRS}^{n+1}
\lessapprox
\ell^{\gamma}_{slab}
\prod_{j=0}^{n} \ell^{2}_{\bBp \in \bBp_{j}} \ell^{2}_{\bMp \subset \bBp \cap slab} \norm{\eit f_{\bBp}}_{L^{p}(\bMp)}
\]
By Corollary~\ref{cor:mixed-norm-mult-Kakeya} with tubes $\bBp$ and boxes $slab$ this is
\[
\lessapprox
\prod_{j=0}^{n} \ell^{2}_{\bBp \in \bBp_{j}} \ell^{p}_{slab} \ell^{2}_{\bMp \subset \bBp \cap slab} \norm{\eit f_{\bBp}}_{L^{p}(\bMp)}
\]
By Theorem~\ref{thm-LRS} and rescaling this is
\[
\lessapprox
\prod_{j=0}^{n} \ell^{2}_{\bBp \in \bBp_{j}} \norm{f_{\bBp}}_{2}
\]
By orthogonality \eqref{eq:orthogonality}, this is
\[
\lessapprox
\prod_{j=0}^{n} \norm{f_{j}}_{2}.
\]
\end{proof}


\section{Fractal Strichartz}
\label{sec-pf}
Somewhat confusingly, we change $p$ to $q$ and use $p$ for a different exponent.

We use boxes and tubes of the following dimensions.
\begin{center}
\begin{tabular}{cccc} \toprule
  Name & Symbol & Size before scaling & Size after scaling\\ \midrule
  Big & $\bB$ & $R$ & \\
  Medium & $\bM$ & $R^{1/2}$ & \\
  Small & $\bS$ & $K^{2}$ & \\
  Big at previous scale & $\bBp$ & $R/K \times R$ & $R/K^{2} = R_{1}$\\
  Medium at previous scale & $\bMp$ & $R^{1/2} \times R^{1/2}K$ & $R^{1/2}/K = R_{1}^{1/2}$\\
  Small at previous scale & $\bSp$ & $KK_{1}^{2} \times K^{2}K_{1}^{2}$ & $K_{1}^{2}$\\ \midrule
  Outer & $\bO$ & $R/r \times R$ & \\
  Inner & $\bI$ & $K^{2} r^{1-4\delta} \times K^{2} r^{2-4\delta}$ & \\
  Outer at previous scale& $\bOp$ & $R/(Kr) \times R$ & $R/(K^{2}r) \times R/K^{2}$\\
  Inner at previous scale & $\bIp$ & $K K_{1}^{2} r^{1-4\delta} \times K^{2} K_{1}^{2} r^{2-4\delta}$ & $K_{1}^{2} r^{1-4\delta} \times K_{1}^{2} r^{2-4\delta}$\\
  \bottomrule
\end{tabular}
\end{center}
The following possible inclusions will be relevant: $\bB \supset \bBp \supset \bMp \supset \bM \supset \bSp \supset \bS$ (for the latter inclusion it is important that $\delta \leq 1/4$).

\begin{proposition} \label{prop:fractal-strichartz}
Let $n\geq 1$.
Let $p=\frac{2(n+1)}{n-1}$ ($p=\infty$ when $n=1$).
Define $s$ and $\kappa$ by $\frac{1}{2} - \frac{1}{s} = \frac{n}{(n+1)(n+2)} = n \kappa$.

Fix $\delta \in (0,1/4)$.
For $R\geq 1$ let $\FrStr(R)$ denote the smallest constant such that the following holds with $K=R^{\delta}$.

For every non-negative weight function $(w_{\bS})$ and every function $f$ with $\supp \widehat{f}\subset B^n(0,1)$ we have
\begin{equation}\label{eq-main}
\ell^{2}_{\bM \subset \bB} \ell^{s}_{\bS \subset \bM} w_{\bS} \norm{ \eit f }_{L^{p}(\bS)}
\leq
\FrStr(R) W \norm{f}_2.
\end{equation}
with
\[
W := \sup_{1 \leq r \leq R^{1/2}} r^{-2\kappa} \sup_{\bO} \Bigl( \sum_{\bI \subset \bO} \bigl( \sum_{\bS \subset \bI} w_{\bS}^{1/(2\kappa)} \bigr)^{2}  \Bigr)^{\kappa}.
\]
Then
\[
\FrStr(R) \lessapprox K^{C} + \FrStr(R/K^{2}).
\]
\end{proposition}
\begin{corollary} \label{cor:fractal-strichartz}
For every $\epsilon>0$ there exists (sufficiently small) $\delta \in (0,1/4)$ such that
\[
\FrStr(R) \lesssim R^{\epsilon}.
\]
\end{corollary}
\begin{remark}
Note that if
\begin{equation}\label{ga}
\gamma:=
\max_{x'\in \R^{n+1},r\geq K^2} r^{-\alpha} \sum_{\bS \subset B(x',r)} w_{\bS}^{1/(2\kappa)}
\end{equation}
then for a fixed $1 \leq r \leq R^{1/2}$ we have
\begin{multline*}
r^{-2} \sup_{\bO} \sum_{\bI \subset \bO} \bigl( \sum_{\bS \subset \bI} w_{\bS}^{1/(2\kappa)} \bigr)^{2}
\leq
r^{-2} \sup_{\bO} \sum_{\bI \subset \bO} \bigl( \sum_{\substack{B \subset \bI\\ \text{ball of radius } K^{2}r^{1-4\delta}}} \sum_{\bS \subset B} w_{\bS}^{1/(2\kappa)} \bigr)^{2}\\
\lesssim
r^{-1} \sup_{\bO} \sum_{\bI \subset \bO} \sum_{\substack{B \subset \bI\\ \text{ball of radius } K^{2}r^{1-4\delta}}} \bigl( \sum_{\bS \subset B} w_{\bS}^{1/(2\kappa)} \bigr)^{2}\\
\lesssim
r^{-1} \sup_{\bO} \sum_{\substack{B \subset \bO\\ \text{ball of radius } K^{2}r^{1-4\delta}}} \bigl( \sum_{\bS \subset B} w_{\bS}^{1/(2\kappa)} \bigr)^{2}\\
\lesssim
\sup_{B_{0} \text{ ball of radius } R/r} \sum_{\substack{B \subset B_{0}\\ \text{ball of radius } K^{2}r^{1-4\delta}}} \bigl( \sum_{\bS \subset B} w_{\bS}^{1/(2\kappa)} \bigr)^{2}\\
\lesssim
\sup_{B_{0}} \sum_{B \subset B_{0}} \bigl( \sum_{\bS \subset B} w_{\bS}^{1/(2\kappa)} \bigr)
\cdot \sup_{B \subset B_{0}} \bigl( \sum_{\bS \subset B} w_{\bS}^{1/(2\kappa)} \bigr)\\
\leq
\bigl( \sup_{B_{0}} \sum_{\bS \subset B_{0}} w_{\bS}^{1/(2\kappa)} \bigr)
\cdot \bigl( \sup_{B} \sum_{\bS \subset B} w_{\bS}^{1/(2\kappa)} \bigr)\\
\lessapprox
\gamma (R/r)^{\alpha} \gamma r^{\alpha}
=
\gamma^{2} R^{\alpha},
\end{multline*}
and taking supremum on the left-hand side we obtain
\[
W \lesssim (\gamma^{2} R^{\alpha})^{\frac{1}{(n+1)(n+2)}}.
\]
Hence Proposition~\ref{prop:fractal-strichartz} implies \cite[Proposition 3.1]{arxiv:1805.02775}.
\end{remark}

\begin{remark}
In the case $w_{\bS} = 1$ for all $\bS \subset \bB$ Proposition~\ref{prop:fractal-strichartz} is weaker than Theorem~\ref{thm-LRS}.
Indeed, in this case
\[
W \sim R^{1/(n+2)},
\]
\[
\ell^{2}_{\bM}
\leq R^{\frac{1}{2} (\frac{1}{2}-\frac{1}{\tilde{p}})} \ell^{\tilde{p}}_{Slab} \ell^{2}_{\bM \subset Slab},
\]
\[
\ell^{s}_{\bS \subset \bM} \leq
R^{\frac{n+1}{2} (\frac{1}{s}-\frac{1}{\tilde{p}})} \ell^{\tilde{p}}_{\bS \subset \bM},
\]
where $\tilde{p} = \frac{2(n+2)}{n}$, and
\[
\frac{1}{2} (\frac{1}{2}-\frac{1}{\tilde{p}})
=
\frac{n+1}{2} (\frac{1}{s}-\frac{1}{\tilde{p}})
=
\frac{1}{2(n+2)}.
\]
\end{remark}

\subsection{Preliminaries}

In the frequency space we decompose $B^n(0,1)$ into disjoint $K^{-1}$-cubes $\tau$.
Denote the set of $K^{-1}$-cubes $\tau$ by $\calS$.
For function $f$ with $\supp \widehat{f}\subset B^n(0,1)$ we have $f=\sum_\tau f_\tau$, where $\widehat{f_\tau}$ is $\widehat{f}$ restricted to $\tau$.
Given a $K^2$-cube $\bS$, we define its \textbf{significant} set
\[
\calS(\bS):=\Set[\Big]{\tau \in \calS \given \norm{\eit f_\tau}_{L^p(\bS)} \geq \frac{1}{100(\#\calS)}\norm{\eit f}_{L^p(\bS)} }\,.
\]
Note that
\[
\norm[\big]{\sum_{\tau\in \calS(B)}\eit f_\tau}_{L^p(\bS)} \sim \norm{\eit f}_{L^p(\bS)}\,.
\]
We say a $K^2$-cube $\bS$ is \textbf{narrow} if there is an $n$-dimensional subspace $V$ such that for all $\tau \in \calS(\bS)$
\[
{\rm Angle}(G(\tau),V) \leq \frac{1}{100nK}\,,
\]
where $G(\tau)\subset \bbS^n$ is a spherical cap of radius $\sim K^{-1}$ and ${\rm Angle}(G(\tau),V)$ denotes the smallest angle between any non-zero vector $v\in V$ and $v'\in G(\tau)$.
Otherwise we say the $K^2$-cube $\bS$ is \textbf{broad}.
It follows from this definition that for any broad $\bS$, there exist $\tau_1,\cdots \tau_{n+1} \in \calS(\bS)$ such that for any $v_j \in G(\tau_j)$
\begin{equation} \label{eq-trans}
\abs{v_1 \wedge v_2\wedge \cdots \wedge v_{n+1}} \gtrsim K^{-n}\,.
\end{equation}

\subsection{Broad case} \label{sec-br}
Let $\beta = 2 \frac{(n+2)(n+1)}{(n+3)n}$, $q = \frac{2(n+2)}{n}$, $s$.
We record a few relations between exponents $2 < \beta < s < q < p$:
\begin{multline*}
\frac{1}{s} - \frac{1}{q} = \frac{1}{q} - \frac{1}{p} = \frac{1}{2} - \frac{1}{\beta} = \frac{1}{(n+1)(n+2)},
\quad
\frac{1}{\beta} - \frac{1}{p} = \frac{1}{2} - \frac{1}{q} = \frac{1}{n+2},\\
\frac{1}{\beta} - \frac{1}{q} = \frac{1}{2} - \frac{1}{s} = \frac{n}{(n+1)(n+2)},
\quad
\frac{1}{2} - \frac{1}{p} = \frac{1}{n+1}.
\end{multline*}

We begin by estimating the contribution of broad cubes.
By H\"older's inequality we estimate
\begin{multline*}
\ell^{2}_{\bM} \ell^{s}_{\bS \subset \bM} w_{\bS} \norm{ \eit f }_{L^{p}(\bS)}
\leq
\ell^{2}_{\bM} (\ell^{1/\kappa}_{\bS \subset \bM} w_{\bS}) \ell^{q}_{\bS \subset \bM} \norm{\eit f}_{L^p(\bS)}\\
\leq
(\ell^{1/\kappa}_{\bM} \ell^{1/\kappa}_{\bS \subset \bM} w_{\bS})
\ell^{\beta}_{\bM} \ell^{q}_{\bS \subset \bM} \norm{\eit f}_{L^p(\bS)}\\
\lessapprox
W
\ell^{\beta}_{\bM} \ell^{q}_{\bS \subset \bM} \norm{\eit f}_{L^p(\bS)},
\end{multline*}
where we used definition of $W$ with $r=1$.
Denote the collection of $(n+1)$-tuple of transverse caps by $\Gamma$:
\[
\Gamma:=\Set[\Big]{ \tilde\tau=(\tau_1,\cdots,\tau_{n+1}) \given \tau_j\in \calS\text{ and } \eqref{eq-trans} \text{ holds for any } v_j\in G(\tau_j) }\,.
\]
Then $\abs{\Gamma} \lesssim K^{O(1)}$.
For some $f_{j} = f_{\tau_{j}}$ with $(\tau_1,\cdots,\tau_{n+1})\in \Gamma$ we have
\[
\norm{ \eit f }_{L^{p}(\bS)}
\lesssim
K^{O(1)} \prod_{j=1}^{n+1} \norm{ \eit f_{j} }_{L^{p}(\bS)}^{1/(n+1)}
\lesssim
K^{O(1)} \prod_{j=1}^{n+1} \norm{ \eit f_{j} }_{L^{q}(\bS)}^{1/(n+1)},
\]
where we used Bernstein's inequality.
By multilinear refined Strichartz we have
\begin{multline*}
\ell^{\beta}_{\bM} \ell^{q}_{\bS \subset \bM} \norm{ \eit f }_{L^{p}(\bS)}
\lesssim
K^{O(1)}
\ell^{\beta}_{\bM} \ell^{q}_{\bS \subset \bM} \prod_{j=1}^{n+1} \norm{ \eit f_{j} }_{L^{q}(\bS)}^{1/(n+1)}\\
\lesssim
K^{O(1)}
\ell^{\beta}_{\bM} \prod_{j=1}^{n+1} \bigl( \ell^{q}_{\bS \subset \bM} \norm{ \eit f_{j} }_{L^{q}(\bS)} \bigr)^{1/(n+1)}\\
\leq
K^{O(1)}
\ell^{\beta}_{\bM} \prod_{j=1}^{n+1} \norm{ \eit f_{j} }_{L^{q}(\bM)}^{1/(n+1)}\\
\lessapprox
\norm{f}_{2}.
\end{multline*}
So the broad case is done.

\subsection{Narrow case} \label{sec-nr}

For each narrow ball, we have the following lemma which is a consequence of $\ell^2$ decoupling theorem in dimension $n$ and Minkowski's inequality.
This argument is essentially contained in Bourgain-Demeter's proof of the $\ell^2$ decoupling conjecture and we omit the details (see the proof of \cite[Proposition 5.5]{MR3374964}).

\begin{lemma}\label{lem-nr-dec}
Suppose that $\bS$ is a narrow $K^2$-cube in $\R^{n+1}$.
Then for any $\epsilon>0$,
\[
\norm{\eit f}_{L^p(\bS)} \leq C_\epsilon K^{\epsilon} \left(\sum_{\tau\in \calS}\norm{\eit f_\tau}_{L^p(\omega_\bS)}^2\right)^{1/2} \,,
\]
here $p=\frac{2(n+1)}{n-1}$, $\calS$ denotes the set of $K^{-1}$-cubes which tile $B^n(0,1)$, and $\omega_\bS$ is a weight function which is essentially a characteristic function on $\bS$.
\end{lemma}

For each $\tau \in \calS$, we will deal with $\eit f_\tau$ by parabolic rescaling and induction on radius.
In order to do so, we need to further decompose $f$ in physical space and perform dyadic pigeonholing several times to get the right picture for our inductive hypothesis at scale $R_1:=R/K^2$ after rescaling.

We do the wave packet decomposition $f = \sum_{\bBp} f_{\bBp}$ with parameter $K$.
For a fixed $\tau$, the different boxes $\bBp_{\tau, B}$ tile $\bB$.
In particular, for each $\tau$, a given $K^2$-cube $\bS$ lies in exactly one box $\bBp_{\tau, D}$.
We write $f=\sum_{\bBp} f_{\bBp}$ for abbreviation.
By Lemma~\ref{lem-nr-dec} and ignoring tails, the narrow contribution is
\begin{multline} \label{eq-dec}
\ell^{2}_{\bM} \ell^{s}_{\bS \subset \bM} w_{\bS} \norm{ \eit f }_{L^{p}(\bS)}
\lessapprox
\ell^{2}_{\bM} \ell^{s}_{\bS \subset \bM} \ell^{2}_{\bBp \supset \bM} w_{\bS} \norm{ \eit f_{\bBp} }_{L^{p}(\bS)}\\
\lessapprox
\ell^{2}_{\bBp} \ell^{2}_{\bM \subset \bBp} \ell^{s}_{\bS \subset \bM} w_{\bS} \norm{ \eit f_{\bBp} }_{L^{p}(\bS)}.
\end{multline}

Fixing $\bBp$ we obtain
\begin{multline}
\label{eq:fixed-Box-norm}
\ell^{2}_{\bM \subset \bBp} \ell^{s}_{\bS \subset \bM} w_{\bS} \norm{ \eit f_{\bBp} }_{L^{p}(\bS)}
\lesssim
\ell^{2}_{\bMp \subset \bBp} K^{1/2-1/s} \ell^{s}_{\bM \subset \bMp} \ell^{s}_{\bSp \subset \bM} w_{\bSp} \ell^{p}_{\bS \subset \bSp} \norm{ \eit f_{\bBp} }_{L^{p}(\bS)}\\
\lesssim
K^{1/2 - 1/s} \ell^{2}_{\bMp \subset \bBp} \ell^{s}_{\bSp \subset \bMp} w_{\bSp} \norm{ \eit f_{\bBp} }_{L^{p}(\bSp)},
\end{multline}
where
\[
w_{\bSp} := \ell^{1/(2\kappa)}_{\bS \subset \bSp} w_{\bS}
\]

Now we use parabolic rescaling to turn the box $\bBp$ into an $R_{1}$-cube.
For each $1/K$-cube $\tau=\tau_{\bBp}$ in $B^n(0,1)$, we write $\xi=\xi_0+K^{-1} \zeta \in \tau$, where $\xi_0$ is the center of $\tau$, then
\[
\abs{\eit f_{\bBp} (x)} =K^{-n/2} \abs{e^{i\tilde t \Delta} g (\tilde x)}
\]
for some function $g$ with Fourier support in the unit cube and $\norm{g}_2=\norm{f_{\bBp}}_2$, where the new coordinates $(\tilde x,\tilde t)$ are related to the old coordinates $(x,t)$ by
\begin{equation} \label{coord}
\begin{cases}
\tilde x =K^{-1} x + 2 t K^{-1} \xi_0\,, \\
\tilde t = K^{-2} t \,.
\end{cases}
\end{equation}
For simplicity, denote the above relation by $(\tilde x,\tilde t)=F(x,t)$.
Therefore
\begin{equation} \label{Ybox-Ytilde}
\norm{\eit f_{\bBp} (x)}_{L^p(\bSp)}
=
K^{\frac{n+2}{p}-\frac{n}{2}} \norm{e^{i\tilde t \Delta} g(\tilde x)}_{L^p(\tilde \bSp)}
=
K^{-\frac{1}{n+1}} \norm{e^{i\tilde t \Delta} g(\tilde x)}_{L^p(\tilde \bSp)},
\end{equation}
where $\tilde \bSp$ is the image of $\bSp$ under the new coordinates.

Note that we can apply our inductive hypothesis \eqref{eq-main} at scale $R_1=R/K^2$ to $\norm{e^{i\tilde t \Delta} g(\tilde x)}_{L^p(\tilde Y)}$ with new weights $w_{\bSp}$.
This gives
\[
\eqref{eq:fixed-Box-norm}
\lessapprox
\underbrace{K^{1/2-1/s} K^{-\frac{1}{n+1}} W_{1}}_{(*)}
\norm{f_{\bBp}}_{2}.
\]
Since $\ell^{2}_{\bBp} \norm{f_{\bBp}}_{2} \lessapprox \norm{f}_{2}$, it remains to observe
\begin{equation}
\label{eq:big-const}
(*)
=
K^{-2\kappa} W_{1}
\lesssim
W.
\end{equation}
Indeed, every $r$ in the definition of $W_{1}$ corresponds to $Kr$ in the definition of $W$.

This finishes the estimate in the narrow case and completes the proof of Proposition~\ref{prop:fractal-strichartz}.

\begin{remark}
Essentially the same argument shows that
\begin{equation}\label{eq:fractal-strichartz:alpha=1}
\ell^{\beta}_{\bM} \ell^{q}_{\bS \subset \bM} w_{\bS} \norm{ \eit f }_{L^{p}(\bS)}
\lessapprox
\sup_{K^{2} \leq r \leq R^{1/2}} \sup_{\bI} \bigl( r^{-2} \sum_{\bS \subset \bI} w_{\bS}^{1/\kappa} \bigr)^{\kappa}
\norm{f}_{2},
\end{equation}
which also implies \cite[Proposition 3.1]{arxiv:1805.02775} when $\alpha=1$ (and is in fact a stronger estimate).
It would be interesting to combine this with Proposition~\ref{prop:fractal-strichartz}.
\end{remark}

\section{Maximal Strichartz}
\label{sec:implications-of-fractal-strichartz}
First we forget the medium scale in Corollary~\ref{cor:fractal-strichartz}.
By H\"older's inequality we have
\[
\ell^{2}_{\bS \subset \bM} w_{\bS} \norm{\eit f}_{L^{p}(\bS)}
\leq
\bigl( \sum_{\bS \subset \bM} w_{\bS}^{1/(1/2-1/s)} \bigr)^{\frac{1}{2}-\frac{1}{s}} \cdot
\ell^{s}_{\bS \subset \bM} w_{\bS} \norm{\eit f}_{L^{p}(\bS)}
\]
With
\[
\lambda := \sup_{\bM} \sum_{\bS \subset \bM} w_{\bS}^{1/(1/2-1/s)}
\]
we obtain
\[
\ell^{2}_{\bS \subset \bB} w_{\bS} \norm{ \eit f }_{L^{p}(\bS)}
=
\ell^{2}_{\bM \subset \bB} \ell^{2}_{\bS \subset \bM} w_{\bS} \norm{ \eit f }_{L^{p}(\bS)}
\lessapprox
\lambda^{n \kappa} W \norm{f}_2.
\]
This recovers \cite[Theorem 1.5]{arxiv:1805.02775}.
If $w$ is a characteristic function, then $\lambda \leq W^{1/(2\kappa)}$, so this implies
\[
\ell^{2}_{\bS \subset \bB} w_{\bS} \norm{ \eit f }_{L^{p}(\bS)}
\lessapprox
W^{n/2+1} \norm{f}_2.
\]
This recovers \cite[Corollary 1.6]{arxiv:1805.02775}.

In particular we also recover \cite[Theorem 2.2]{arxiv:1805.02775} which in turn improves \cite[Theorem 1.7]{MR3842310}: if $\mu$ is an $\alpha$-dimensional measure and $f$ has Fourier support in the unit ball, then
\[
\norm{\sup_{0 < t < R} \abs{\eit f}}_{L^{2}(B(0,R),\dif\mu)}
\lessapprox
R^{\frac{\alpha}{2(n+1)}} \norm{f}_{2}.
\]
The most optimistic conjecture for $\alpha=n$ would be
\[
\norm{\sup_{0 < t < R} \abs{\eit f}}_{L^{2(n+1)/n}(B(0,R))}
\lessapprox
\norm{f}_{2},
\]
which would imply the sharp $L^{2}$ estimate by H\"older's inequality.
This is known for $n=1$ \cite{MR1101221} and $n=2$ \cite{MR3702674}.
However, this conjecture is false for $n \geq 3$ by \cite{arxiv:1902.01430}.

On the other hand, if $\mu$ is an $\alpha$-dimensional measure with $\frac{3n+1}{4} \leq \alpha \leq n$, then \cite{MR3574661} \cite{arxiv:1703.01360} show that pointwise convergence of solutions fails for $s < \frac{n + (n-1)(n-\alpha)}{2(n+1)}$.
I do not trust \cite{MR3574661} because the union in (2.7) of that paper is not disjoint, so the lower bound for $\abs{\Omega}$ does not seem to hold.

\cite{MR2264734}: reduction of $H^{s}$ estimates to local estimates.

\section{Using multilinear restriction}
\begin{theorem}[{Multilinear restriction \cite{MR2275834}}] \label{thm:mult-restr}
In the setting of Theorem~\ref{thm-kRS} we have
\begin{equation} \label{kRS2}
\ell^{2(n+1)/n}_{\bS \subset \bB} \prod_{j=0}^{n} \norm{ \eit f_{j} }_{L^{2(n+1)/n}(\bS)}^{1/(n+1)}
\lessapprox \nu^{-C}
\prod_{j=0}^{n} \norm{f_{j}}_{2}^{1/(n+1)}.
\end{equation}
\end{theorem}

\begin{proposition} \label{prop:fractal-strichartz2}
Let $2 \leq t \leq 2(n+1)/n$, $p_{\mathrm{mr},n}=2(n+1)/n$, $p_{\mathrm{dec},n-1}=2(n+1)/(n-1)$,
\[
\frac{1}{t_{1}} = \frac{1}{t} - \frac{1}{p_{\mathrm{mr},n}}, \quad
\frac{1}{t_{2}} = \frac{1}{t} - \frac{1}{p_{\mathrm{dec},n-1}}, \quad
\frac{1}{t_{3}} = \frac{n+2}{p_{\mathrm{dec},n-1}} - \frac{n}{2}.
\]
In the setting of Proposition~\ref{prop:fractal-strichartz} let now $\FrStr(R)$ denote the smallest constant such that the following holds with $K=R^{\delta}$.

For every non-negative weight function $(w_{\bS})$ and every function $f$ with $\supp \widehat{f}\subset B^n(0,1)$ we have
\begin{equation}\label{eq-main2}
\ell^{t}_{\bS \subset \bB} w_{\bS} \norm{ \eit f }_{L^{p}(\bS)}
\leq
\FrStr(R) W \norm{f}_2.
\end{equation}
with
\[
W := \sup_{1 \leq r \leq R^{1/2}} r^{1/t_{3}} \sup_{\bO} \ell^{t_{1}}_{\bI \subset \bO} \ell^{t_{2}}_{\bS \subset \bI} w_{\bS}.
\]
Then
\[
\FrStr(R) \lessapprox K^{C} + \FrStr(R/K^{2}).
\]
\end{proposition}
\begin{proof}
In the broad case
\[
\norm{ \eit f }_{L^{p}(\bS)}
\lesssim
K^{O(1)} \prod_{j=1}^{n+1} \norm{ \eit f_{j} }_{L^{p}(\bS)}^{1/(n+1)}
\lesssim
K^{O(1)} \prod_{j=1}^{n+1} \norm{ \eit f_{j} }_{L^{2(n+1)/n}(\bS)}^{1/(n+1)},
\]
so
\begin{multline*}
\ell^{t}_{\bS \subset \bB} w_{\bS} \norm{ \eit f }_{L^{p}(\bS)}
\lesssim
K^{O(1)} \ell^{t}_{\bS \subset \bB} w_{\bS} \prod_{j=1}^{n+1} \norm{ \eit f_{j} }_{L^{2(n+1)/n}(\bS)}^{1/(n+1)}\\
\leq
K^{O(1)} \ell^{t_{1}}_{\bS \subset \bB} w_{\bS} \cdot \ell^{2(n+1)/n} \prod_{j=1}^{n+1} \norm{ \eit f_{j} }_{L^{2(n+1)/n}(\bS)}^{1/(n+1)}\\
\lessapprox
\ell^{t_{1}}_{\bS \subset \bB} w_{\bS} \cdot \prod_{j=0}^{n} \norm{f_{j}}_{2}^{1/(n+1)}\\
\leq
\ell^{t_{1}}_{\bS \subset \bB} w_{\bS} \cdot \norm{f}_{2}.
\end{multline*}
In the narrow case by decoupling
\begin{multline}
\ell^{t}_{\bS \subset \bB} w_{\bS} \norm{ \eit f }_{L^{p}(\bS)}
\lessapprox
\ell^{t}_{\bS \subset \bB} \ell^{2}_{\bBp \supset \bS} w_{\bS} \norm{ \eit f_{\bBp} }_{L^{p}(\bS)}\\
\lessapprox
\ell^{2}_{\bBp} \ell^{t}_{\bS \subset \bBp} w_{\bS} \norm{ \eit f_{\bBp} }_{L^{p}(\bS)}.
\end{multline}
Fixing $\bBp$ we obtain
\begin{multline} \label{eq:fixed-Box-norm2}
\ell^{t}_{\bS \subset \bBp} w_{\bS} \norm{ \eit f_{\bBp} }_{L^{p}(\bS)}
\lesssim
\ell^{t}_{\bSp \subset \bBp} \ell^{t}_{\bS \subset \bSp} w_{\bS} \norm{ \eit f_{\bBp} }_{L^{p}(\bS)}\\
\leq
\ell^{t}_{\bSp \subset \bBp} \ell^{t_{2}}_{\bS \subset \bSp} w_{\bS} \cdot \ell^{p}_{\bS \subset \bSp} \norm{ \eit f_{\bBp} }_{L^{p}(\bS)}\\
\lesssim
\ell^{t}_{\bSp \subset \bBp} w_{\bSp} \cdot \norm{ \eit f_{\bBp} }_{L^{p}(\bSp)},
\end{multline}
where
\[
w_{\bSp} := \ell^{t_{2}}_{\bS \subset \bSp} w_{\bS}.
\]
Parabolic rescaling gives
\[
\eqref{eq:fixed-Box-norm2}
\lessapprox
\underbrace{K^{-1/(n+1)} W_{1}}_{(*)}
\norm{f_{\bBp}}_{2},
\]
where $W_{1}$ is computed using the new weights $w_{\bSp}$.
Since $\ell^{2}_{\bBp} \norm{f_{\bBp}}_{2} \lessapprox \norm{f}_{2}$, it remains to observe
\begin{equation}
\label{eq:big-const2}
(*)
=
K^{-1/(n+1)} W_{1}
\lesssim
W.
\end{equation}
Indeed, every $r$ in the definition of $W_{1}$ corresponds to $Kr$ in the definition of $W$.

This finishes the estimate in the narrow case and completes the proof of Proposition~\ref{prop:fractal-strichartz2}.
\end{proof}

Note that $t_{2} < t_{1}$, and it follows that
\begin{multline*}
W
=
\sup_{1 \leq r \leq R^{1/2}} r^{-1/(n+1)} \sup_{\bO} \Bigl( \sum_{\bI \subset \bO} \bigl( \sum_{\bS \subset \bI} w_{\bS}^{t_{2}} \bigr)^{t_{1}/t_{2}}  \Bigr)^{1/t_{1}}\\
\leq
\sup_{1 \leq r \leq R^{1/2}} r^{-1/(n+1)} \sup_{\bO} \Bigl( \sum_{\bI \subset \bO} \bigl( \sum_{\bS \subset \bI} w_{\bS}^{t_{2}} \bigr) \cdot \sup_{\bI \subset \bO} \bigl( \sum_{\bS \subset \bI} w_{\bS}^{t_{2}} \bigr)^{t_{1}/t_{2}-1}  \Bigr)^{1/t_{1}}\\
\leq
\sup_{1 \leq r \leq R^{1/2}} r^{-1/(n+1)} \sup_{\bO} \Bigl( \sum_{\bS \subset \bO} w_{\bS}^{t_{2}} \Bigr)^{1/t_{1}} \cdot \sup_{\bI \subset \bO} \bigl( \sum_{\bS \subset \bI} w_{\bS}^{t_{2}} \bigr)^{1/t_{2}-1/t_{1}}
\end{multline*}
If $w$ is the characteristic function of an $\alpha$-dimensional collection, then this is
\begin{multline*}
\lessapprox
\sup_{1 \leq r \leq R^{1/2}} r^{1/t_{3}} \Bigl( r (R/r)^{\alpha} \Bigr)^{1/t_{1}} \cdot \bigl( r r^{\alpha} \bigr)^{1/t_{2}-1/t_{1}}\\
=
R^{\alpha/t_{1}} \sup_{1 \leq r \leq R^{1/2}} r^{1/t_{2}+1/t_{3}} r^{\alpha(1/t_{2}-2/t_{1})}
=
R^{\alpha/t_{1}} \sup_{1 \leq r \leq R^{1/2}} r^{1/t-1/2} r^{\alpha(1/2-1/t)}\\
=
R^{\alpha/t_{1}} \sup_{1 \leq r \leq R^{1/2}} r^{(\alpha-1)(1/2-1/t)}
=
R^{\alpha/t_{1}} R^{(\alpha-1)(1/2-1/t)/2},
\end{multline*}
assuming $\alpha \geq 1$.
For the $\ell^{2}_{\bS \subset \bB}$ norm this gives the estimate
\[
W \ell^{t_{3}}_{\bS \subset \bB} w_{\bS},
\]
where $1/t_{3} = 1/2 - 1/t$.
This is
\begin{multline*}
\leq
R^{\alpha/t_{1}} R^{(\alpha-1)(1/2-1/t)/2} R^{\alpha/t_{3}}
=
R^{\alpha/(2(n+1))} R^{(\alpha-1)(1/2-1/t)/2}.
\end{multline*}
In particular, for $n=\alpha=1$ we recover the $L^{4}$ Schr\"odinger maximal estimate with $t=4$.
\end{document}
