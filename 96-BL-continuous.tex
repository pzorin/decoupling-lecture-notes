\documentclass[biblatex]{pzorin-note}
\usepackage{amsmath}
\usepackage{amssymb}
\usepackage{amsthm}
\usepackage{xspace}
\usepackage{mathtools}
\usepackage{etoolbox}
\usepackage{todonotes}

% Companion to \to. Example:
%  Let $ f \from A \to B $ be a function.
\newcommand*{\from}{\colon}


\newcommand*{\Z}{\mathbb{Z}}
\newcommand*{\Q}{\mathbb{Q}}
\def\C{\mathbb{C}}
\newcommand*{\N}{\mathbb{N}}
\newcommand*{\R}{\mathbb{R}}
\newcommand*{\K}{\mathbb{K}}
\newcommand*{\E}{\mathbb{E}}
\newcommand*{\T}{\mathbb{T}}
\newcommand*{\Rp}{\mathbb{R}_+}
\newcommand*{\boundary}{\partial}
\newcommand*{\id}{\mathrm{id}}
\newcommand*{\Id}{\mathrm{Id}}
\newcommand*{\Quot}{\mathrm{Quot}}
\newcommand*{\Gal}{\mathrm{Gal}}
\newcommand*{\hol}[1]{\mathcal{H}(#1)}
\newcommand*{\Rang}[1]{\mathrm{Rang}\left(#1\right)}
\newcommand*{\Mat}{\mathrm{Mat}}
\newcommand*{\End}{\mathrm{End}}
\newcommand*{\Aut}{\mathrm{Aut}}
\newcommand*{\const}{\mathrm{const}}
\newcommand*{\z}{\bar z}
\newcommand*{\inv}{^{-1}}
\newcommand*{\cconv}{\overline{\mathrm{conv}}\,}
\newcommand*{\Union}{\bigcup\limits}
\newcommand*{\Intersection}{\bigcap\limits}
\newcommand*{\union}{\cup}
\newcommand*{\intersection}{\cap}
\newcommand*{\Sum}{\sum\limits}
\newcommand*{\Product}[1]{\prod\limits_{#1}}
\newcommand*{\Prod}{\prod\limits}
\newcommand*{\Meet}{\bigwedge\limits}
\newcommand*{\JOIN}{\bigvee\limits}
\newcommand*{\join}{\vee}
\newcommand*{\meet}{\wedge}
\newcommand*{\with}{\,:\,}
\newcommand*{\unit}[1]{\,\mathrm{#1}}
\newcommand*{\BMO}{\mathrm{BMO}}
\newcommand*{\VMO}{\mathrm{VMO}}
\newcommand*{\Schwartz}{\mathcal{S}}
\newcommand{\one}{\mathbf{1}}
\newcommand*{\widevec}[1]{\overrightarrow{#1}}

% Differential d's
\newcommand{\dif}{\mathop{}\!\mathrm{d}} % \mathop produces two thin spaces, \! removes the trailing one
\newcommand*{\tdif}[3][]{\frac{\dif^{ #1} #2}{\dif { #3}^{ #1}}}
\newcommand*{\ttdif}[2]{\frac{\dif^2 #1}{\dif {#2} ^2}}
\newcommand*{\pdif}[3][]{\frac{\partial^{ #1} #2}{\partial { #3}^{ #1}}}
\newcommand*{\ppdif}[2]{\frac{\partial^{2} #1}{\partial {#2}^{2}}}

\let\sphi\phi % Stroked phi
\renewcommand*{\phi}{\varphi}
\renewcommand*{\epsilon}{\varepsilon}

\def\<{\left\langle}
\def\>{\right\rangle}

% DeclareMathOperator without the repetition
\newcommand*{\DMO}[1]{\expandafter\DeclareMathOperator\csname #1\endcsname {#1}}

\DMO{card}
\DMO{Ran}
\DMO{diam}
\DMO{dist}
\DMO{dom}
\DMO{Ad}
\DMO{ad}
\DMO{Shift}
\DMO{esssup}
\DMO{tr}
\DMO{grad}
\DMO{Kern}
\DMO{Bild}
\DMO{Lin}
\DMO{Hom}
\DMO{lin}
\DMO{sg}
\DMO{sign}
\DMO{ord}
\DMO{supp}
\DMO{ggT}
\DMO{kgV}
\DMO{Res}
\DMO{Ann}
\DMO{Ass}
\DMO{Rad}
\DMO{conv}
\DMO{im}
\DMO{lcm}
\DMO{rank}

% Theorem-like environments
\theoremstyle{plain}
\ifdefined\theorem\else % Already defined by beamer
\newtheorem{theorem}{Theorem}
\newtheorem{thm}[theorem]{Theorem}
\newtheorem{proposition}[theorem]{Proposition}
\newtheorem{prop}[theorem]{Proposition}
\newtheorem{lemma}[theorem]{Lemma}
\newtheorem{lem}[theorem]{Lemma}
\newtheorem{corollary}[theorem]{Corollary}
\newtheorem{cor}[theorem]{Corollary}
\newtheorem{conjecture}[theorem]{Conjecture}
\ifdefined\section % May be not defined in scrlttr2
\numberwithin{equation}{section}
\numberwithin{theorem}{section}
\else\fi
\fi

\theoremstyle{remark}
\ifdefined\remark\else % Already defined by beamer
\newtheorem*{remark}{Remark}
\newtheorem*{claim}{Claim}
\fi
\newtheorem*{speculation}{Speculation}
\newtheorem*{prob}{Problem}
\ifdefined\example\else % Already defined by beamer
\newtheorem*{example}{Example}
\fi
\ifdefined\question\else % Already defined by beamer
\newtheorem*{question}{Question}
\fi

\theoremstyle{definition}
\ifdefined\definition\else % Already defined by beamer
\newtheorem{definition}[theorem]{Definition}
\fi
\newtheorem{defi}[theorem]{Definition}

% Proper names with accents
\newcommand*{\Frechet}{Fr\'echet\xspace} % Fréchet
\newcommand*{\Konig}{K\H{o}nig\xspace} % Kőnig
\newcommand*{\Cesaro}{Ces\`aro\xspace} % Cesàro
\newcommand*{\Folner}{F\o{}lner\xspace} % Følner

% Absolute values and norms using mathtools. \[lr][vV]ert produces correct spacing as opposed to | and \|.
\DeclarePairedDelimiter\abs{\lvert}{\rvert}
\DeclarePairedDelimiter\norm{\lVert}{\rVert}
\DeclarePairedDelimiter\floor{\lfloor}{\rfloor}
\DeclarePairedDelimiter\ceil{\lceil}{\rceil}
\DeclarePairedDelimiterX\innerp[2]{\langle}{\rangle}{#1,#2}
% just to make sure it exists
\providecommand\given{}
% can be useful to refer to this outside \Set
\newcommand\SetSymbol[1][]{%
\nonscript\:#1\vert
\allowbreak
\nonscript\:
\mathopen{}}
\DeclarePairedDelimiterX\Set[1]\{\}{%
\renewcommand\given{\SetSymbol[\delimsize]}
#1
}

% Alexander Perlis | A complement to \smash, \llap, and \rlap
% math.arizona.edu/~aprl/publications/mathclap/
% For comparison, the existing overlap macros:
% \def\llap#1{\hbox to 0pt{\hss#1}}
% \def\rlap#1{\hbox to 0pt{#1\hss}}
\def\clap#1{\hbox to 0pt{\hss#1\hss}}
\def\mathllap{\mathpalette\mathllapinternal}
\def\mathrlap{\mathpalette\mathrlapinternal}
\def\mathclap{\mathpalette\mathclapinternal}
\def\mathllapinternal#1#2{%
\llap{$\mathsurround=0pt#1{#2}$}}
\def\mathrlapinternal#1#2{%
\rlap{$\mathsurround=0pt#1{#2}$}}
\def\mathclapinternal#1#2{%
\clap{$\mathsurround=0pt#1{#2}$}}

\def\PZdefchar#1{
  \expandafter\def\csname frak#1\endcsname{\mathfrak{#1}}
  \expandafter\def\csname bb#1\endcsname{\mathbb{#1}}
  \expandafter\def\csname bf#1\endcsname{\mathbf{#1}}
  \expandafter\def\csname scr#1\endcsname{\mathcal{#1}}
  \expandafter\def\csname cal#1\endcsname{\mathcal{#1}}}
\def\PZdefloop#1{\ifx#1\PZdefloop\else\PZdefchar#1\expandafter\PZdefloop\fi}
\PZdefloop abcdefghijklmnopqrstuvwxyzABCDEFGHIJKLMNOPQRSTUVWXYZ\PZdefloop

% Work around beamer's \pause and amsmath's align incompatibility
% https://tex.stackexchange.com/a/75550/30158
\makeatletter
\let\save@measuring@true\measuring@true
\def\measuring@true{%
  \save@measuring@true
  \def\beamer@sortzero##1{\beamer@ifnextcharospec{\beamer@sortzeroread{##1}}{}}%
  \def\beamer@sortzeroread##1<##2>{}%
  \def\beamer@finalnospec{}%
}
\makeatother

\def\PZdefchar#1{
  \expandafter\def\csname frak#1\endcsname{\mathfrak{#1}}
  \expandafter\def\csname bf#1\endcsname{\mathbf{#1}}
  \expandafter\def\csname scr#1\endcsname{\mathcal{#1}}
  \expandafter\def\csname cal#1\endcsname{\mathcal{#1}}}
\def\PZdefloop#1{\ifx#1\PZdefloop\else\PZdefchar#1\expandafter\PZdefloop\fi}
\PZdefloop abcdefghijklmnopqrstuvwxyzABCDEFGHIJKLMNOPQRSTUVWXYZ\PZdefloop

\def\tA{\tilde{A}}
\def\tD{\tilde{D}}
\def\tV{\tilde{V}}
\def\tGamma{\tilde{\Gamma}}
\def\tgamma{\tilde{\gamma}}
\def\C{\mathbb{C}}

\DeclarePairedDelimiterXPP\EE[1]{\E}{\lparen}{\rparen}{}{\renewcommand\given{\SetSymbol[\delimsize]}#1} % Conditional expectation \EE{ f \given A }

\makeatletter
\newcommand\@avsum[2]{%
  {\sbox0{$\m@th#1\sum$}%
   \vphantom{\usebox0}%
   \ooalign{%
     \hidewidth
     \smash{\vrule height\dimexpr\ht0+1pt\relax depth\dimexpr\dp0+1pt\relax}%
     \hidewidth\cr
     $\m@th#1\sum$\cr
   }%
  }%
}
\newcommand{\avsum}{\mathop{\mathpalette\@avsum\relax}\displaylimits}
\newcommand\@avprod[2]{%
  {\sbox0{$\m@th#1\prod$}%
   \vphantom{\usebox0}%
   \ooalign{%
     \hidewidth
     \smash{\vrule height\dimexpr\ht0+1pt\relax depth\dimexpr\dp0+1pt\relax}%
     \hidewidth\cr
     $\m@th#1\prod$\cr
   }%
  }%
}
\newcommand{\avprod}{\mathop{\mathpalette\@avprod\relax}\displaylimits}
\newcommand{\@avL}[2]{%
\ooalign{{$\m@th#1\mbox{--}$}\cr {$\m@th#1 L$}\cr}}
\newcommand{\avL}{\mathpalette\@avL\relax}
\newcommand{\@avell}[2]{%
\ooalign{{$\m@th#1\mbox{--}$}\cr {$\m@th#1 \ell$}\cr}}
\newcommand{\avell}{\mathpalette\@avell\relax}
\newcommand{\@avD}{%
  \ooalign{{$\mathrm{D}$}\cr \hidewidth\raise.2ex\hbox{$\vert$}\hidewidth\cr}}
\newcommand{\avDec}{\@avD\mathrm{ec}}
\makeatother
\newcommand{\Dec}{\mathrm{Dec}}
\newcommand{\MulDec}{\mathrm{MulDec}}
\newcommand{\RvSq}{\mathrm{RvSq}}
\newcommand{\MlRvSq}{\mathrm{MlRvSq}}
\newcommand{\BL}{\mathrm{BL}}
\newcommand{\BLg}{\mathrm{BL}_{\mathbf{g}}}
\newcommand{\ED}{E^{\calD}}
\newcommand{\Part}[2][]{\calP(\ifstrempty{#1}{}{#1,}#2)} % Partition of #1 at scale #2.
\newcommand{\Cov}[2][\R^{d+n}]{\calB(#1,#2)} % Covering of #1 at scale #2.

\usepackage{esint}
\newcommand{\MK}{\mathcal{K}}
\newcommand{\Tubes}{\mathbf{T}}
\def\FT{\mathcal{F}}

\usepackage{tikz}
% This code defines TikZ "parabola through" path operation, taken from
% https://tex.stackexchange.com/a/429938/30158
% In that post it is explained that TikZ builtin "parabola" is wrong.
\makeatletter
\def\pt@get#1#2{
  \tikz@scan@one@point\pgfutil@firstofone#2\relax%
  \csname pgf@x#1\endcsname=\pgf@x%
  \csname pgf@y#1\endcsname=\pgf@y%
}
\tikzset{
  parabola through/.style={
    to path={{[x={(\pgf@xc,\pgf@yc)}, y=\parabola@y, shift=(\tikztostart)]
      -- (0,0) .. controls (1/3,1/3) and (2/3,1/3) .. (1,0) \tikztonodes}--(\tikztotarget)}
  },
  parabola through/.prefix code={
    \pt@get{a}{(\tikztostart)}\pt@get{b}{#1}\pt@get{c}{(\tikztotarget)}%
    \advance\pgf@xb by-\pgf@xa\advance\pgf@yb by-\pgf@ya%
    \advance\pgf@xc by-\pgf@xa\advance\pgf@yc by-\pgf@ya%
    \pgfmathsetmacro\parabola@y{(\pgf@yc-\pgf@xc/\pgf@xb*\pgf@yb)%
      /(\pgf@xb-\pgf@xc)*\pgf@xc}%
  }
}
\makeatother

\numberwithin{equation}{section}
\theoremstyle{definition}
\newtheorem{hypothesis}[equation]{Hypothesis}

\renewcommand*{\det}{\qopname\relax o{det}} % equivalent to \nolimits

\begin{document}

\subsection{Continuity of BL constants}
Need extremizers, simple proof \cite{MR2661170} not enough.

In this section we present the argument from \cite{MR3723636} that shows that the gaussian BL constant \eqref{BLfunctional} is continuous in the tuple of linear maps.
An earlier and weaker result that would also suffice for our purposes was proved in \cite{MR3783217}.
For \emph{simple} BL data a more precise result is available, namely that the BL constant is differentiable \cite{MR2836590}.
A more precise result for not necessarily simple BL data was proved in \cite{arxiv:1811.11052}, namely that the BL constant is locally H\"older.

We parameterize positive definite matrices by a rotation matrix and a diagonal matrix of their (positive) eigenvalues.
To shorten double sums of the form $\sum_{j=1}^m \sum_{i = 1}^{n_j}$ we define
\[
\calK := \Set{ (j,i) \given 1 \leq j \leq m, 1 \leq i \leq n_{j} }.
\]
For $(i,j) \in \calK$ let $p_{(j,i)} := p_{j}$.
We write $\calI$ for the set of all subsets $I$ of $\calK$ of cardinality $n$.
For each $I\in\calI$ define
\[
p_I := \prod_{k\in I} p_k,
\quad
\lambda_I := \prod_{k\in I} \lambda_k
\]
for any family $(\lambda_{k})_{k\in\calK} \subset (0,\infty)$.
\begin{lemma}
\label{thm:BL-lambda-R}
For suitable nonnegative nonnegative continuous functions $d_{I}$, $I\in\calI$, we have
\begin{equation}
\label{eq:BL-lambda-R}
\BL_{\bfg}(\bfB,\bfp)^{2} = \sup \Set*{
\frac{\prod_{k\in\calK} \lambda_k^{p_k} }{ \sum_{I\in \calI}  \lambda_I p_I d_I(\bfB,\bfR) } \given \lambda_k \in (0,\infty), R_i \in \mathrm{SO}(n_i)
},
\end{equation}
where we denote by $\bfR$ the $m$-tuple $(R_i)_{i=1}^m$.
\end{lemma}
\begin{proof}
Recall from \eqref{gauss},
\[
\BL_{\bfg}(\bfB,\bfp)^{2}
=
\sup_{A_{1},\dotsc,A_{m}} \;\frac{\prod_{j=1}^m(\det A_j)^{p_j}}{\det\left(\sum_{j=1}^m p_jL_{j}^*A_jL_{j}\right)}.
\]
Here $A_j$ is a positive definite $n_j \times n_j$ matrix.
We fix an orthonormal basis $\Set{ \vec{e}_{j,i} \given i = 1, \dots, n_j }$ for each $H_{j} = \R^{n_j}$ and parameterize $A_j = R_j^* D_j R_j$, where $R_j \in \mathrm{SO}(n_i)$ is a rotation matrix and $D_j = \sum_{i=1}^{n_j} \lambda_{j,i} \vec{e}_{j,i} ({\vec{e}_{j,i}})^*$ is a diagonal matrix with positive diagonal entries $\lambda_{j,1}, \dotsc, \lambda_{j,n_j}$.
Using the notation introduced above,
\[
\prod_{j=1}^m (\det A_j)^{p_j}
=
\prod_{k\in\calK} \lambda_k^{p_k}.
\]
On the other hand,
\begin{align*}
\sum_{j=1}^m p_jL_{j}^*A_jL_{j}
& =
\sum_{j=1}^m p_jL_{j}^*R_j^*\left(  \sum_{i=1}^{n_j} \lambda_{j,i} \vec{e}_{j,i}(\vec{e}_{j,i})^* \right)   R_jL_{j}
\\ & =
\sum_{(j,i) \in \calK} p_j \lambda_{j,i} L_{j}^* R_j^* \vec{e}_{j,i} \left(L_{j}^* R_j^* \vec{e}_{j,i} \right)^*
\\ & =
\sum_{k \in \calK} p_k \lambda_k v_k  v_k^*
=: T
\end{align*}
with $v_{j,i} := L_{j}^* R_j^* \vec{e}_{j,i}$
We compute $\det(T)$ using the Cauchy--Binet formula.
Define the $n \times \abs{\calK}$ matrices
\[
A = (\lambda_k  p_k v_k)_{k \in \calK},
\quad
B = (v_{k})_{k \in \calK}.
\]
For $I \in \calI$ define $n \times n$ matrices
\[
A_{I} = (\lambda_k  p_k v_k)_{k \in I},
\quad
B_{I} = (v_{k})_{k \in I}.
\]
Then
\[
\det(T)=\det(AB^{*})
= \sum_{I\in \calI} \det (A_I B_I^{*})
= \sum_{I\in \calI} \det(B_{I}) \left( \prod_{k \in I} \lambda_k  p_k \right) \det (B_I^{*})
= \sum_{I\in \calI}  \lambda_I p_I d_I
\]
with
\[
d_I
=
d_{I}(\bfB,\bfR)
:=
\det (B_I)^{2}
=
\det((v_k)_{k\in I})^2.
\qedhere
\]
\end{proof}

\begin{lemma}
\label{lem:BL-Barthe-fct-cont}
The function $F : [0,\infty)^{\calI} \to [0,+\infty]$ given by
\begin{equation}
\label{eq:BL-Barthe-formula}
F(\bfd)
:=
\sup_{\lambda_k>0} \frac{ \prod_{k\in\calK} \lambda_k^{p_k}  }{ \sum_{I \in \calI} d_I p_I \lambda_I }
\end{equation}
is continuous.
\end{lemma}

\begin{proof}
First, \eqref{eq:BL-Barthe-formula} is lower semicontinuous as a supremum of lower semicontinuous functions.
It remains to prove upper semicontinuity. 

Fix a point $\widetilde{\bfd}\in\R^N$.
Suppose that $\bfd\in\R^N$ is such that
\[
\abs{d_{I}-\widetilde{d}_{I}} \leq \delta \abs{\widetilde{d}_{I}}
\text{ for those } I \text{ for which } \widetilde{d}_{I} \neq 0
\]
for some $\delta>0$.
Then
\begin{align*}
F(\bfd)
&=
\sup_{\lambda_k>0} \frac{ \prod_{k\in\calK} \lambda_k^{p_k} }{ \sum_{\mathcal{I}} d_I p_I\lambda_I}
\\ &\leq
\sup_{\lambda_k>0} \frac{ \prod_{k\in\calK} \lambda_k^{p_k} }{ \sum_{\mathcal{I}} \widetilde{d}_I(1-\delta) p_I\lambda_I},
\\ &=
F(\tilde{\bfd}) ( 1-\delta )^{-1}.
\end{align*}
This proves the required upper semicontinuity.
\end{proof}

\begin{theorem}
\label{thm:gauss-BL-continuous}
For each $\bfp$,  the map $\bfB \mapsto \BL_{\bfg}(\bfB,\bfp)$ is a continuous function with values in $[0,+\infty]$.
\end{theorem}

\begin{proof}[Proof of Theorem~\ref{thm:gauss-BL-continuous}]
By Lemma~\ref{thm:BL-lambda-R} we have
\[
\BL_{\bfg}(\bfB,\bfp)^{2} = \sup \Set*{
F(\bfd(\bfB,\bfR)) \given  R_i \in \mathrm{SO}(n_i) },
\]
where $F$ is defined by \eqref{eq:BL-Barthe-formula}.
By Lemma~\ref{lem:BL-Barthe-fct-cont} and construction of $\bfd$ the function $(\bfB,\bfR) \mapsto F(\bfd(\bfB,\bfR))$ is continuous, hence locally uniformly continuous.
Since the supremum is taken over a compact set, it is also continuous.
\end{proof}

\end{document}



