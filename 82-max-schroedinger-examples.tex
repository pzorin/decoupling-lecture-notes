\documentclass[biblatex]{pzorin-note}
\usepackage{amsmath}
\usepackage{amssymb}
\usepackage{amsthm}
\usepackage{xspace}
\usepackage{mathtools}
\usepackage{etoolbox}
\usepackage{todonotes}

% Companion to \to. Example:
%  Let $ f \from A \to B $ be a function.
\newcommand*{\from}{\colon}


\newcommand*{\Z}{\mathbb{Z}}
\newcommand*{\Q}{\mathbb{Q}}
\def\C{\mathbb{C}}
\newcommand*{\N}{\mathbb{N}}
\newcommand*{\R}{\mathbb{R}}
\newcommand*{\K}{\mathbb{K}}
\newcommand*{\E}{\mathbb{E}}
\newcommand*{\T}{\mathbb{T}}
\newcommand*{\Rp}{\mathbb{R}_+}
\newcommand*{\boundary}{\partial}
\newcommand*{\id}{\mathrm{id}}
\newcommand*{\Id}{\mathrm{Id}}
\newcommand*{\Quot}{\mathrm{Quot}}
\newcommand*{\Gal}{\mathrm{Gal}}
\newcommand*{\hol}[1]{\mathcal{H}(#1)}
\newcommand*{\Rang}[1]{\mathrm{Rang}\left(#1\right)}
\newcommand*{\Mat}{\mathrm{Mat}}
\newcommand*{\End}{\mathrm{End}}
\newcommand*{\Aut}{\mathrm{Aut}}
\newcommand*{\const}{\mathrm{const}}
\newcommand*{\z}{\bar z}
\newcommand*{\inv}{^{-1}}
\newcommand*{\cconv}{\overline{\mathrm{conv}}\,}
\newcommand*{\Union}{\bigcup\limits}
\newcommand*{\Intersection}{\bigcap\limits}
\newcommand*{\union}{\cup}
\newcommand*{\intersection}{\cap}
\newcommand*{\Sum}{\sum\limits}
\newcommand*{\Product}[1]{\prod\limits_{#1}}
\newcommand*{\Prod}{\prod\limits}
\newcommand*{\Meet}{\bigwedge\limits}
\newcommand*{\JOIN}{\bigvee\limits}
\newcommand*{\join}{\vee}
\newcommand*{\meet}{\wedge}
\newcommand*{\with}{\,:\,}
\newcommand*{\unit}[1]{\,\mathrm{#1}}
\newcommand*{\BMO}{\mathrm{BMO}}
\newcommand*{\VMO}{\mathrm{VMO}}
\newcommand*{\Schwartz}{\mathcal{S}}
\newcommand{\one}{\mathbf{1}}
\newcommand*{\widevec}[1]{\overrightarrow{#1}}

% Differential d's
\newcommand{\dif}{\mathop{}\!\mathrm{d}} % \mathop produces two thin spaces, \! removes the trailing one
\newcommand*{\tdif}[3][]{\frac{\dif^{ #1} #2}{\dif { #3}^{ #1}}}
\newcommand*{\ttdif}[2]{\frac{\dif^2 #1}{\dif {#2} ^2}}
\newcommand*{\pdif}[3][]{\frac{\partial^{ #1} #2}{\partial { #3}^{ #1}}}
\newcommand*{\ppdif}[2]{\frac{\partial^{2} #1}{\partial {#2}^{2}}}

\let\sphi\phi % Stroked phi
\renewcommand*{\phi}{\varphi}
\renewcommand*{\epsilon}{\varepsilon}

\def\<{\left\langle}
\def\>{\right\rangle}

% DeclareMathOperator without the repetition
\newcommand*{\DMO}[1]{\expandafter\DeclareMathOperator\csname #1\endcsname {#1}}

\DMO{card}
\DMO{Ran}
\DMO{diam}
\DMO{dist}
\DMO{dom}
\DMO{Ad}
\DMO{ad}
\DMO{Shift}
\DMO{esssup}
\DMO{tr}
\DMO{grad}
\DMO{Kern}
\DMO{Bild}
\DMO{Lin}
\DMO{Hom}
\DMO{lin}
\DMO{sg}
\DMO{sign}
\DMO{ord}
\DMO{supp}
\DMO{ggT}
\DMO{kgV}
\DMO{Res}
\DMO{Ann}
\DMO{Ass}
\DMO{Rad}
\DMO{conv}
\DMO{im}
\DMO{lcm}
\DMO{rank}

% Theorem-like environments
\theoremstyle{plain}
\ifdefined\theorem\else % Already defined by beamer
\newtheorem{theorem}{Theorem}
\newtheorem{thm}[theorem]{Theorem}
\newtheorem{proposition}[theorem]{Proposition}
\newtheorem{prop}[theorem]{Proposition}
\newtheorem{lemma}[theorem]{Lemma}
\newtheorem{lem}[theorem]{Lemma}
\newtheorem{corollary}[theorem]{Corollary}
\newtheorem{cor}[theorem]{Corollary}
\newtheorem{conjecture}[theorem]{Conjecture}
\ifdefined\section % May be not defined in scrlttr2
\numberwithin{equation}{section}
\numberwithin{theorem}{section}
\else\fi
\fi

\theoremstyle{remark}
\ifdefined\remark\else % Already defined by beamer
\newtheorem*{remark}{Remark}
\newtheorem*{claim}{Claim}
\fi
\newtheorem*{speculation}{Speculation}
\newtheorem*{prob}{Problem}
\ifdefined\example\else % Already defined by beamer
\newtheorem*{example}{Example}
\fi
\ifdefined\question\else % Already defined by beamer
\newtheorem*{question}{Question}
\fi

\theoremstyle{definition}
\ifdefined\definition\else % Already defined by beamer
\newtheorem{definition}[theorem]{Definition}
\fi
\newtheorem{defi}[theorem]{Definition}

% Proper names with accents
\newcommand*{\Frechet}{Fr\'echet\xspace} % Fréchet
\newcommand*{\Konig}{K\H{o}nig\xspace} % Kőnig
\newcommand*{\Cesaro}{Ces\`aro\xspace} % Cesàro
\newcommand*{\Folner}{F\o{}lner\xspace} % Følner

% Absolute values and norms using mathtools. \[lr][vV]ert produces correct spacing as opposed to | and \|.
\DeclarePairedDelimiter\abs{\lvert}{\rvert}
\DeclarePairedDelimiter\norm{\lVert}{\rVert}
\DeclarePairedDelimiter\floor{\lfloor}{\rfloor}
\DeclarePairedDelimiter\ceil{\lceil}{\rceil}
\DeclarePairedDelimiterX\innerp[2]{\langle}{\rangle}{#1,#2}
% just to make sure it exists
\providecommand\given{}
% can be useful to refer to this outside \Set
\newcommand\SetSymbol[1][]{%
\nonscript\:#1\vert
\allowbreak
\nonscript\:
\mathopen{}}
\DeclarePairedDelimiterX\Set[1]\{\}{%
\renewcommand\given{\SetSymbol[\delimsize]}
#1
}

% Alexander Perlis | A complement to \smash, \llap, and \rlap
% math.arizona.edu/~aprl/publications/mathclap/
% For comparison, the existing overlap macros:
% \def\llap#1{\hbox to 0pt{\hss#1}}
% \def\rlap#1{\hbox to 0pt{#1\hss}}
\def\clap#1{\hbox to 0pt{\hss#1\hss}}
\def\mathllap{\mathpalette\mathllapinternal}
\def\mathrlap{\mathpalette\mathrlapinternal}
\def\mathclap{\mathpalette\mathclapinternal}
\def\mathllapinternal#1#2{%
\llap{$\mathsurround=0pt#1{#2}$}}
\def\mathrlapinternal#1#2{%
\rlap{$\mathsurround=0pt#1{#2}$}}
\def\mathclapinternal#1#2{%
\clap{$\mathsurround=0pt#1{#2}$}}

\def\PZdefchar#1{
  \expandafter\def\csname frak#1\endcsname{\mathfrak{#1}}
  \expandafter\def\csname bb#1\endcsname{\mathbb{#1}}
  \expandafter\def\csname bf#1\endcsname{\mathbf{#1}}
  \expandafter\def\csname scr#1\endcsname{\mathcal{#1}}
  \expandafter\def\csname cal#1\endcsname{\mathcal{#1}}}
\def\PZdefloop#1{\ifx#1\PZdefloop\else\PZdefchar#1\expandafter\PZdefloop\fi}
\PZdefloop abcdefghijklmnopqrstuvwxyzABCDEFGHIJKLMNOPQRSTUVWXYZ\PZdefloop

% Work around beamer's \pause and amsmath's align incompatibility
% https://tex.stackexchange.com/a/75550/30158
\makeatletter
\let\save@measuring@true\measuring@true
\def\measuring@true{%
  \save@measuring@true
  \def\beamer@sortzero##1{\beamer@ifnextcharospec{\beamer@sortzeroread{##1}}{}}%
  \def\beamer@sortzeroread##1<##2>{}%
  \def\beamer@finalnospec{}%
}
\makeatother

\def\PZdefchar#1{
  \expandafter\def\csname frak#1\endcsname{\mathfrak{#1}}
  \expandafter\def\csname bf#1\endcsname{\mathbf{#1}}
  \expandafter\def\csname scr#1\endcsname{\mathcal{#1}}
  \expandafter\def\csname cal#1\endcsname{\mathcal{#1}}}
\def\PZdefloop#1{\ifx#1\PZdefloop\else\PZdefchar#1\expandafter\PZdefloop\fi}
\PZdefloop abcdefghijklmnopqrstuvwxyzABCDEFGHIJKLMNOPQRSTUVWXYZ\PZdefloop

\def\tA{\tilde{A}}
\def\tD{\tilde{D}}
\def\tV{\tilde{V}}
\def\tGamma{\tilde{\Gamma}}
\def\tgamma{\tilde{\gamma}}
\def\C{\mathbb{C}}

\DeclarePairedDelimiterXPP\EE[1]{\E}{\lparen}{\rparen}{}{\renewcommand\given{\SetSymbol[\delimsize]}#1} % Conditional expectation \EE{ f \given A }

\makeatletter
\newcommand\@avsum[2]{%
  {\sbox0{$\m@th#1\sum$}%
   \vphantom{\usebox0}%
   \ooalign{%
     \hidewidth
     \smash{\vrule height\dimexpr\ht0+1pt\relax depth\dimexpr\dp0+1pt\relax}%
     \hidewidth\cr
     $\m@th#1\sum$\cr
   }%
  }%
}
\newcommand{\avsum}{\mathop{\mathpalette\@avsum\relax}\displaylimits}
\newcommand\@avprod[2]{%
  {\sbox0{$\m@th#1\prod$}%
   \vphantom{\usebox0}%
   \ooalign{%
     \hidewidth
     \smash{\vrule height\dimexpr\ht0+1pt\relax depth\dimexpr\dp0+1pt\relax}%
     \hidewidth\cr
     $\m@th#1\prod$\cr
   }%
  }%
}
\newcommand{\avprod}{\mathop{\mathpalette\@avprod\relax}\displaylimits}
\newcommand{\@avL}[2]{%
\ooalign{{$\m@th#1\mbox{--}$}\cr {$\m@th#1 L$}\cr}}
\newcommand{\avL}{\mathpalette\@avL\relax}
\newcommand{\@avell}[2]{%
\ooalign{{$\m@th#1\mbox{--}$}\cr {$\m@th#1 \ell$}\cr}}
\newcommand{\avell}{\mathpalette\@avell\relax}
\newcommand{\@avD}{%
  \ooalign{{$\mathrm{D}$}\cr \hidewidth\raise.2ex\hbox{$\vert$}\hidewidth\cr}}
\newcommand{\avDec}{\@avD\mathrm{ec}}
\makeatother
\newcommand{\Dec}{\mathrm{Dec}}
\newcommand{\MulDec}{\mathrm{MulDec}}
\newcommand{\RvSq}{\mathrm{RvSq}}
\newcommand{\MlRvSq}{\mathrm{MlRvSq}}
\newcommand{\BL}{\mathrm{BL}}
\newcommand{\BLg}{\mathrm{BL}_{\mathbf{g}}}
\newcommand{\ED}{E^{\calD}}
\newcommand{\Part}[2][]{\calP(\ifstrempty{#1}{}{#1,}#2)} % Partition of #1 at scale #2.
\newcommand{\Cov}[2][\R^{d+n}]{\calB(#1,#2)} % Covering of #1 at scale #2.

\usepackage{esint}
\newcommand{\MK}{\mathcal{K}}
\newcommand{\Tubes}{\mathbf{T}}
\def\FT{\mathcal{F}}

\usepackage{tikz}
% This code defines TikZ "parabola through" path operation, taken from
% https://tex.stackexchange.com/a/429938/30158
% In that post it is explained that TikZ builtin "parabola" is wrong.
\makeatletter
\def\pt@get#1#2{
  \tikz@scan@one@point\pgfutil@firstofone#2\relax%
  \csname pgf@x#1\endcsname=\pgf@x%
  \csname pgf@y#1\endcsname=\pgf@y%
}
\tikzset{
  parabola through/.style={
    to path={{[x={(\pgf@xc,\pgf@yc)}, y=\parabola@y, shift=(\tikztostart)]
      -- (0,0) .. controls (1/3,1/3) and (2/3,1/3) .. (1,0) \tikztonodes}--(\tikztotarget)}
  },
  parabola through/.prefix code={
    \pt@get{a}{(\tikztostart)}\pt@get{b}{#1}\pt@get{c}{(\tikztotarget)}%
    \advance\pgf@xb by-\pgf@xa\advance\pgf@yb by-\pgf@ya%
    \advance\pgf@xc by-\pgf@xa\advance\pgf@yc by-\pgf@ya%
    \pgfmathsetmacro\parabola@y{(\pgf@yc-\pgf@xc/\pgf@xb*\pgf@yb)%
      /(\pgf@xb-\pgf@xc)*\pgf@xc}%
  }
}
\makeatother

\numberwithin{equation}{section}
\theoremstyle{definition}
\newtheorem{hypothesis}[equation]{Hypothesis}

\renewcommand*{\det}{\qopname\relax o{det}} % equivalent to \nolimits

\begin{document}
This is a copy of \cite{arxiv:1703.01360}.

\section{Introduction}
Consider the Schr\"odinger equation, $i \partial_{t} u +\Delta u=0$, in $\R^{n+1}$, with
initial data $u(\cdot, 0) = u_0$ in the Bessel potential/Sobolev space defined by
\[
H^{s}(\R^{n}) = (1-\Delta)^{-s/2}L^2(\R^{n}):=\left\{
G_{s} * g \, : \, g \in L^{2}(\R^n)
\right\} \, .
\]
The Bessel kernel $G_{s}$ is defined as usual via its Fourier transform; $\widehat{G}_{s} = (1 + \abs{\cdot}^{2})^{- s/2}$.
In~\cite{Carl}, Carleson proposed the problem of identifying
the exponents $s > 0$ for which
\begin{equation}\label{CarlesonProblem}
\lim_{t\to 0} u(x , t) = u_0(x), \qquad \text{a.e.} \quad x \in \R^n, \qquad \forall \ u_0\in H^s(\R^{n}) \, ,
\end{equation}
with respect to Lebesgue measure, and proved that \eqref{CarlesonProblem} holds as long as
$s \geq 1/4$ and $n=1$. Dahlberg and Kenig then showed that this condition is necessary, providing a complete solution in the one-dimensional case~\cite{DahlKenig}.



In higher dimensions, \eqref{CarlesonProblem} holds as long as
$s > \frac{2n-1}{4n}$; see \cite{L, B}.
It was thought that $s\ge 1/4$ might also be sufficient in higher dimensions (see for example \cite{GS} or \cite{T}), however
Bourgain recently proved that~$s\ge\frac{n}{2(n+1)}$ is necessary \cite{Bnew}. Since then, Du, Guth and Li \cite{DGL} improved the sufficient condition in two dimensions to the almost sharp $s>1/3$.

Here we give a new proof of the necessary condition using a different example (fewer frequencies travelling in a skew direction; see \eqref{fg}).
We replace number theoretic arguments, via comparison with Guass sums, with ergodic arguments that exploit
the occasional complete absence of cancelation as in \cite{LuR2}.
This permits us to generalise to fractional Hausdorff measure.
When $n=2$, the proof becomes much simpler as the ergodic arguments are trivial in that case.

\begin{theorem}\label{Thm:DirectConv}
Let $ (3n+1)/4 \leq \alpha \leq n$.
Then, for any
\begin{equation}\label{INTVAL}
s < \frac{n}{2(n+1)}+\frac{n-1}{2(n+1)}(n-\alpha) \, ,
\end{equation}
there exists $u_0 \in H^{s}(\mathbb{R}^{n})$ such that
\[
\limsup_{t \to 0} \abs{ u(x,t) } = \infty
\]
for all $x$ in a set
of positive $\alpha$--Hausdorff measure.
\end{theorem}


The study of this refined version of Carleson's problem was initiated by Sj\"ogren and Sj\"olin \cite{SS}. Theorem~\ref{Thm:DirectConv} improves \cite[Theorem 2]{LuR2}, although the result there holds for the full range $n/2\le \alpha\le n$ (with $\alpha<n/2$ the question was previously resolved in \cite{BBCR}).
It has been conjectured that $s\ge\frac{n}{2(n+1)}$ should also be sufficient in the $\alpha=n$ case; see \cite{DG}. If that were true, then \eqref{INTVAL} would represent the interpolating condition between two sharp results, and so it would be interesting to see if Theorem~\ref{Thm:DirectConv} could be extended to the range $n/2\le \alpha\le n$, or whether there is a discontinuity in behaviour as in the one-dimensional case.

Indeed, defining
\[\alpha_n(s):=\sup_{u_0\in H^s(\R^n)}\mathrm{dim}\Big\{\ x\in\mathbb{R}^n\ :\ \limsup_{t\to0}\abs{u(x,t)}=\infty \ \Big\} \, ,\]
where $\mathrm{dim}$ denotes the Hausdorff dimension, the combination of Theorem~\ref{Thm:DirectConv} with previous results yields
\begin{equation*}
\alpha_n(s)\ge \left \{
\begin{array}{rcccccccl}
&n &\text{when}&\!\!\!\!\!& \!\!&s\!\!&< &\!\!\!\frac{n}{2(n+1)}&\\ [0.8ex]
& n+\frac{n}{n-1}-\frac{2(n+1)s}{n-1} &\text{when}& \frac{n}{2(n+1)}\!\!\!\!\!&\le\!\!& s\!\!&<&\!\!\!\frac{n+1}{8}&\\ [0.8ex]
&n+1-\frac{2(n+2)s}{n} &\text{when}&\frac{n+1}{8}\!\!\!\!\!& \le\!\!& s\!\!&<& \!\!\!\frac{n}{4}&\\ [0.8ex]
& n-2s\qquad &\text{when}&\frac{n}{4} \!\!\!\!\!& \le \!\!& s\!\!&\le& \!\!\!\frac{n}{2}&\!\!\!\!\!\!\!\! \, .
\end{array}\right.
\end{equation*}
The function on the right-hand side is continuous apart from a jump of $\frac{1}{2n}$ over the regularity $s=\frac{n+1}{8}$. The bound is best possible in one dimension, in which case the central intervals are empty and the dimension jumps by a half over $s=1/4$. This is a consequence of the Dahlberg--Kenig example combined with~\cite{BBCR}, where it was proven that $\alpha_n(s)\le n-2s$ in the range $n/4\le s\le n/2$. For the best known upper bounds with lower regularity, see \cite[Theorem 1.2]{LuR}.



In the following section we present the quantitive ergodic lemma that will be used in the third section to provide a new proof that $s\ge\frac{n}{2(n+1)}$ is necessary in the Lebesgue measure case. For this we will employ the Niki\v sin--Stein maximal principle. However, in the fourth section, we will explicitly construct data for which the divergence occurs, see \eqref{TestinfFunctBennFin}, enabling the proof of Theorem~\ref{Thm:DirectConv}.


\section{A quantitive ergodic lemma}

It is well-known that linear flow on the torus, in most directions, eventually passes arbitrarily close to every point. This remains true
when only considering equidistant points on the trajectory.

\begin{lemma}\label{erg}
Let $d \geq 2$, $0<\varepsilon,\delta<1$ and $\kappa>\frac{1}{d+1}$. Then, if $\delta<\kappa$ and $R>1$ is sufficiently large,
there is $\theta \in \mathbb{S}^{d-1}$ for which, given any $y\in \mathbb{T}^{d}$ and $a\in\mathbb{R}$, there is a $t_y\in R^\delta\mathbb{Z}\cap(a,a+R)$ such that
\begin{equation*}
\abs{y-t_y\theta}\le \varepsilon R^{(\kappa-1)/d} \, .
\end{equation*}
Moreover, this remains true with $d=1$, for some $\theta\in(0,1)$.
\end{lemma}

\begin{proof} When $d=1$, by taking $\theta$ close to $R^{-1}$, we obtain approximately~$R^{1-\delta}$ points~$t_y\theta$ equally spaced at intervals of length $R^{\delta-1}$ on the circle. For each $y\in \mathbb{T}$, one of these points $t_y\theta$ must lie closer than a distance of $\varepsilon R^{\kappa-1}$ if $R$ is sufficiently large so that $R^{\delta-\kappa}<\varepsilon$.

When $d\ge 2$ and $a=0$, this was proved in~\cite[Lemma~2]{LuR2}. The adjustment to the general case $a\in\R$ amounts to little more than starting the flow at different points on the translation invariant torus. One can also easily check that the proof in \cite{LuR2} is essentially unchanged. One need only translate their function $\eta_R$ by $a$, and the modulus of the Fourier transform of this is unchanged, so the remainder of the argument is exactly the same.
\end{proof}




The following corollary is optimal, in the sense that the statement fails for larger~$\sigma$.
To see this, we can place balls
of radius~$\varepsilon R^{-\frac{\gamma}{d}}$ centred at the points of the sets below and assume that the balls are disjoint.
Then the volume of such a set would be of the order $\varepsilon^dR^{d+1/2-\gamma-(d+2)\sigma}$, a quantity that
is arbitrarily small for larger $\sigma$.
Neither is it possible to extend the range of~$\gamma$, as then the set of times~$t$ could be empty. To avoid this we must
have $\sigma < 1/4$ which is ensured by the restriction~$\gamma \geq 3d/4$.



\begin{corollary}\label{Corollary:ToroImpr}
Let $d\ge 2$, $\frac{3d}{4}\le \gamma \le d$ and $0< \sigma < \frac{1+2(d- \gamma)}{2(d+2)}$.
Then, for any $\varepsilon>0$ and sufficiently large $R >1$,
there exists $\theta \in \mathbb{S}^{d-1}$ such that
\begin{equation*}
\bigcup_{t\in R^{2\sigma-1}\mathbb{Z}\cap(a,a+R^{-1/2})} \big\{x\in R^{\sigma-1} \mathbb{Z}^{d}\, :\, \abs{x}\le 2\big\}+t\theta
\end{equation*}
is $\varepsilon R^{-\frac{\gamma}{d}}$-dense in $B(0,1/2)$, for all $a\in (0,1)$. Moreover, this remains true with $d=1$, for some $\theta\in(0,1)$.
\end{corollary}




\begin{proof}
We first rescale by $R^{1-\sigma}$,
%, this is equivalent to showing that
%\begin{equation}\nonumber
%\bigcup_{t \in R^{\sigma} \mathbb{Z}\cap(R^{1-\sigma}a,R^{1-\sigma}a+R^{1/2-\sigma})} \big\{x\in \mathbb{Z}^{d}\, :\, \abs{x}\le 2R^{1-\sigma} \big\}+t\theta
%\end{equation}
%is $\varepsilon R^{1-\sigma-\frac{\alpha}{d}}$--dense in $B(0,R^{1-\sigma}/2)$ for a certain $\theta \in \mathbb{S}^{d-1}$.
and then replace $R^{1/2-\sigma}$ by $R$. In this way the statement is equivalent to proving that,
for any $y\in B(0,R^{\frac{1-\sigma}{1/2-\sigma}}/2)$ there exists
\[
x_{y} \in \mathbb{Z}^{d}\cap B(0,2R^\frac{1-\sigma}{1/2-\sigma})\quad \text{and}\quad
t_{y}\in R^{\frac{\sigma}{1/2-\sigma}} \mathbb{Z}\cap (R^{\frac{1-\sigma}{1/2-\sigma}}a,R^{\frac{1-\sigma}{1/2-\sigma}}a+R)
\]
such that
\begin{equation*}\label{EquivTorusImpr}
\abs{y - (x_{y} + t_{y}\theta) }\ < \ \varepsilon R^{\frac{1-\sigma}{1/2-\sigma}-\frac{\gamma}{d(1/2-\sigma)}} \, ,
\end{equation*}
for a fixed $\theta \in \mathbb{S}^{d-1}$, independent of $y$ and $a$.
By taking the quotient $\R^n / \mathbb{Z}^{d} = \mathbb{T}^{d}$, this would follow if, for any $[y] \in \mathbb{T}^{d}$, we have
%there exists \[t_{y} \in R^{\frac{\sigma}{1/2-\sigma}} \mathbb{Z}\cap (R^{\frac{1-\sigma}{1/2-\sigma}}a,R^{\frac{1-\sigma}{1/2-\sigma}}a+R)\] such that
\begin{equation*}
\abs{[y] - [t_{y} \theta]} \ < \ \varepsilon R^{\frac{1-\sigma}{1/2-\sigma}-\frac{\gamma}{d(1/2-\sigma)}} \, .
\end{equation*}
Now this is a consequence of Lemma~\ref{erg}, by taking $\delta=\sigma/(1/2-\sigma)$ and $\kappa$ so that
\[
\frac{\kappa-1}{d}=\frac{1-\sigma}{1/2-\sigma}-\frac{\gamma}{d(1/2-\sigma)} \, .
\]
The conditions $0<\delta<1$ and $\delta < \kappa$ are then ensured by the restrictions on $\gamma$ and
$\sigma$ in the statement.
\end{proof}




\section{Proof of the Lebesgue measure necessary condition}

When the initial data $u_{0}$ is a Schwartz function, the solution $u$ to the Schr\"odinger equation can be written as
\begin{equation*}
u(x, t) = e^{i t \Delta}u_{0}(x):=\frac{1}{(2\pi)^{n/2}}\int_{\R^n}\widehat{u}_{0}(\xi)\, e^{ix\cdot\xi -it \abs{ \xi }^{2}} d \xi \, .
\end{equation*}
By the Niki\v sin--Stein maximal
principle \cite{N, st}, it suffices to prove the following theorem.

\begin{theorem}\label{max2}
Suppose that there is a constant $C_s$ such that
\begin{equation*}
\norm{ \sup_{0< t < 1} \abs[\big]{ e^{it\Delta} f } }_{L^{2}(B(0,1))} \le C_s\norm{ f }_{H^{s}(\R^n)} \, ,
\end{equation*}
whenever $f$ is a Schwartz function. Then $s\ge\frac{n}{2(n+1)}$.
\end{theorem}

\begin{proof}
Writing $t/(2\pi R)$ in place of $t$, the maximal estimate implies that\footnote{We write $a\lesssim b$ ($a\gtrsim b$) whenever $a$ and $b$ are nonnegative quantities that satisfy $a \leq C b$ ($a \geq C b$) for a constant $C > 0$. We write $a\simeq b$ when $a\lesssim b$ and $b\lesssim a$.}
\begin{equation}\label{otooBis0}
\norm{ \sup_{0<t<1} \abs[\big]{ e^{i\frac{t}{2\pi R}\Delta} f } }_{L^{2}(B(0,1))} \lesssim R^s\norm{ f }_{2} \, ,
\end{equation}
whenever $\supp \widehat{f} \subset B(0,2R)$ and $R>4$. From now on we let $B(0,\rho)$ denote the $(n-1)$-dimensional ball of radius $\rho>0$, a fixed, sufficiently small constant.
Writing $x=(x_1,\bar{x})$ and letting $0 < \sigma < \frac{1}{2(n+1)}$, we consider frequencies in the set
\begin{equation}\nonumber
\Omega := \big\{ \bar{\xi}\in 2\pi R^{1-\sigma} \mathbb{Z}^{n-1} \,:\, \abs{\bar{\xi}}\le R \big\} + B(0,\rho) \,,
\end{equation}
and Schwartz functions defined by $\widehat{\phi}=\chi_{(-\rho,\rho)}$ and
$\widehat{g} =\chi_{\Omega}$. Then the initial data is defined by
\begin{equation}\label{fg}
f(x)= e^{i\pi R(1,\theta)\cdot x}\phi(R^{1/2}x_1)g(\bar{x}),
\end{equation}
where $\theta \in (0,1)$ when $n=2$ and $\theta \in \mathbb{S}^{n-2}$ in higher dimensions.

Note that the solution factorises
\begin{equation}\label{eq:factrization}
e^{i\frac{t}{2\pi R}\Delta} f(x) =e^{i\frac{t}{2\pi R}\Delta} f_{dk}(x_1) e^{i\frac{t}{2\pi R}\Delta} f_{\theta}(\bar{x}) \, ,
\end{equation}
where $f_{dk}$ and $f_\theta$ are defined by
\[
f_{dk}(x_1)=e^{i\pi Rx_1}\phi(R^{1/2}x_1)\quad\text{and}\quad f_\theta(\bar{x}) = e^{i\pi R\theta\cdot \bar{x}} g(\bar{x}) \, .
\]
By a change of variables, we have
\begin{align}\label{dk}
\abs{e^{i\frac{t}{2\pi R}\Delta} f_{dk}(x_1)}&=\frac{R^{-1/2}}{(2\pi)^{1/2}}\abs[\Big]{\int \widehat{\phi}\big(R^{-1/2}(\xi_1-\pi R)\big)e^{ ix_1\xi_1 -i\frac{t}{2\pi R} \xi_1^{2}} d \xi_1}\\\nonumber
&=\frac{1}{(2\pi)^{1/2}}\abs[\Big]{\int_{-\rho}^\rho e^{ iR^{1/2}(x_1-t)y-i\frac{t}{2\pi} y^{2}} d y}\simeq 1 \, ,
\end{align}
whenever $\abs{t}\le 1$ and $\abs{x_1-t}\le R^{-1/2}$. %; of course in the second identity we have changed variables $y = R^{-1/2}(\xi_1-\pi R)$.
Indeed, these restrictions ensure that the phase is close to zero, so that no cancelation occurs in the integral.
By Plancherel's identity and Fubini's theorem,
\begin{equation}\nonumber
\norm{f}_2=\norm{f_{dk}}_2\norm{f_\theta}_2\simeq R^{-1/4}\abs{\Omega}^{1/2}\, ,
\end{equation}
so that plugging the data into the maximal estimate \eqref{otooBis0} and using \eqref{eq:factrization} and \eqref{dk},
we obtain
\begin{equation}\label{otooBis}
\Big(\int_{B(0,1)} \int_0^{1/2} \sup_{t\in(x_1,x_1+R^{-1/2})} \abs[\big]{ e^{i\frac{t}{2\pi R}\Delta} f_\theta(\bar{x}) }^2dx_1d\bar{x}\Big)^{1/2} \lesssim R^sR^{-1/4}\abs{\Omega}^{1/2} \, .
\end{equation}

In order to understand the behaviour of $e^{i\frac{t}{2\pi R}\Delta} f_\theta$ we first consider the unmodulated version $e^{i\frac{t}{2\pi R}\Delta} g$.
Barcel\'o, Bennett, Carbery, Ruiz and Vilela \cite{BBCRV} showed that
\begin{equation}\label{Phase=1}
\abs{e^{i\frac{t}{2\pi R}\Delta} g(\bar{x})} \gtrsim \abs{\Omega},
\quad
\mbox{for all}
\quad
(\bar{x},t) \in X_0\times R^{2\sigma-1}\mathbb{Z}\cap(0,1) \, ,
\end{equation}
where, with $\varepsilon$ sufficiently small, $X_0$ is defined by
\begin{equation}\nonumber
X_0 = \big\{ \bar{x}\in R^{\sigma-1}\mathbb{Z}^{n-1}\,:\, \abs{\bar{x}}\le 2\big\}+ B(0,\varepsilon R^{-1}) \, .
\end{equation}
This time the phase in the integrand never strays too far from zero modulo~$2\pi i$, and so again there is no cancelation in the integral. Now
\begin{equation}\nonumber
(\bar{x},t)\in X_{t\theta} \times R^{2\sigma - 1} \mathbb{Z}\cap(0,1) \quad
\Rightarrow\quad (\bar{x}-t \theta,t) \in X_0\times R^{2\sigma-1}\mathbb{Z}\cap(0,1) \, ,
\end{equation}
where $X_{t\theta}:=X_0+t\theta$ and $
\abs[\big]{ e^{i\frac{t}{2\pi R}\Delta} f_\theta(\bar{x})}= \abs[\big]{ e^{i\frac{t}{2\pi R}\Delta} g(\bar{x} -t\theta )}
$.
Combining this fact with (\ref{Phase=1}) yields
\begin{equation*}\label{otooBisr}
\sup_{t\in(x_1,x_1+R^{-1/2})} \abs[\big]{ e^{i\frac{t}{2\pi R}\Delta} f_\theta(\bar{x})} \gtrsim \abs{\Omega},
\quad
\text{for all}\quad
\bar{x} \in \Gamma_{\!x_1} :=\!\!\!\!\bigcup_{t\in R^{2\sigma - 1} \mathbb{Z}\cap(x_1,x_1+ R^{-1/2})}\!\!\!\!X_{t\theta} \,,
\end{equation*} and this holds uniformly for all $x_1\in (0,1/2)$.

Now the sets $\Gamma_{\!x_1}$ can be considered to be $\varepsilon R^{-1}$--neighbourhoods of the sets of Corollary~\ref{Corollary:ToroImpr}. So, taking $\gamma=d=n-1$,
there is a~$\theta\in \mathbb{S}^{n-2}$ so that $B(0,1/2)\subset \Gamma_{\!x_1}$ for
all $x_1\in(0,1/2)$.
Substituting into \eqref{otooBis}, this yields
\[
\abs{\Omega}^{1/2} \lesssim R^sR^{-1/4} \, .
\]
As $\abs{\Omega}\simeq R^{(n-1)\sigma}$, we can let $\sigma$ tend to $\frac{1}{2(n+1)}$ and then $R$ tend to infinity, so that
\[
s\ge 1/4+\frac{n-1}{4(n+1)}=\frac{n}{2(n+1)} \, ,
\]
which completes the proof.
\end{proof}

















\section{Proof of Theorem~\ref{Thm:DirectConv}}


The solution is typically represented as $u(\cdot,t):= \lim_{N \to \infty} S_{N}(t)u_0$, where
\begin{equation}\label{rt}
S_{N}(t)u_0(x) :=
\frac{1}{(2\pi)^{n/2}} \int_{\mathbb{R}^n}
\Psi(N^{-1}\xi)\,
\widehat{u}_0(\xi)\, e^{ ix\cdot\xi - i t \abs{ \xi }^{2} } d \xi \, ,
\end{equation}
and $\Psi$ is a fixed function, equal to one near the origin, that decays in such a way that the integral is well-defined. For
convenience we take $\Psi(\xi)=\prod_{j=1}^{n} \psi(\xi_{j})$, where~$\psi$ is differentiable, supported in the interval $[-2,2]$ and equal to one on $[-1,1]$. The limit is usually taken with respect to the $L^2$--norm, but here we will take all limits pointwise, at each point that they exist. Supposing that $\alpha>n-2s$, as we may, the limits exist at almost every $x$ with respect to $\alpha$--Hausdorff measure (see for example \cite[Corollary 17.6]{M}) and they coincide with the usual $L^2$--limit almost everywhere with respect to Lebesgue measure.




We take $0< \sigma < \frac{1+2(n-\alpha) }{2(n+1)}$
and $\lambda := 2^{\frac{M}{1-\sigma}}$, with $M\in \mathbb{Z}$ to be chosen sufficiently large later.
As $\alpha \geq (3n+1)/4$ we have that $\sigma < 1/4$.
Writing $\xi=(\xi_1,\bar{\xi})$, we consider the sets of frequencies
\begin{equation}\nonumber
\Omega^{j} = \big\{ \bar{\xi} \in 2 \pi \lambda^{j(1-\sigma)} \mathbb{Z}^{n-1} \,:\, \lambda^{j} \leq \abs{ \xi_{m} }< \lambda^{j+1}, \, m=2,\ldots, n \big\} + Q\Big( 0,\frac{\varepsilon_{1}}{\sqrt{n\!-\!1}} \Big) \, ,
\end{equation}
where
$\varepsilon_{1} > 0$ is a fixed sufficiently small constant and $j\in \mathbb{N}$.
Here~$Q(0,\ell)$
is the closed~$(n-1)$-dimensional cube centred at the origin with side-length $\ell$, and we denote its interior by $\mathring{Q}(0,\ell)$.
For a suitable choice of $\theta_{j} \in (0,1)$ when $n=2$ or $\theta_{j}\in \mathbb{S}^{n-2}$ in higher dimensions, the initial data $u_0$ that gives rise to a divergent solution is given by
\begin{equation}\label{TestinfFunctBennFin}
u_0(x) := \sum_{j \in \mathbb{N}} e^{i\pi \lambda^j(1,\theta_{j})\cdot x}\phi(\lambda^{j/2}x_1)g_{j}(\bar{x}).
\end{equation}
Here $\widehat{\phi}=\chi_{(-\varepsilon_{1},\varepsilon_{1})}$ and
$\widehat{g}_{j} := \lambda^{j \delta } \abs{\Omega^j}^{-1}\chi_{\Omega^{j}}$, with $0< \delta< \sigma/4$.
Noting that $\abs{ \Omega^{j}} \simeq \lambda^{j (n -1)\sigma}$, we have that $u_0\in H^{s}$ whenever
\begin{equation*}\label{URONS}
s < \frac{(n-1) \sigma }{2}+\frac{1}{4} - \delta \, .
\end{equation*}
Eventually we will let $\sigma$ tend to $ \frac{1+2(n-\alpha) }{2(n+1)}$ and $\delta$ tend to zero, covering all the cases of the range \eqref{INTVAL}.

First we consider $(n-1)$-dimensional data given by $f_{\theta_{j}}(\bar{x}) := e^{ i\pi \lambda^{j} \theta_{j} \cdot \bar{x} }g_{j}(\bar{x})$ and the associated solutions on $X_{t\theta_{j}}^{j} \times
T^{j}_{x_1}$ defined by
\begin{equation*}
X_{t\theta_j}^{j}= \{ \bar{x} \in \lambda^{j(\sigma-1)}\mathbb{Z}^{n-1} \, : \, \abs{\bar{x}} \leq 2 \} + \mathring{Q} ( t\theta_{j}, \varepsilon_{2} \lambda^{-j} ) \, ,
\end{equation*}
\[
T^{j}_{x_1} = \big\{ t\in \lambda^{j(2\sigma-1)}\mathbb{Z}\, : \, x_1<t<x_1+\lambda^{-j/2}\big\} \, .
\]
Taking $\varepsilon_{1}$, $\varepsilon_{2}$ sufficiently small and $x_1\in(0,1/2)$, as in the previous section we have
\begin{equation}\label{GalInvPreq}
\abs[\Big]{ S_{N} \Big( \frac{t}{2\pi \lambda^{j}} \Big) f_{\theta_{j}}(\bar{x}) } \gtrsim \lambda^{j \delta},
\quad
\mbox{for all}
\quad
(\bar{x}, t) \in X^{j}_{t\theta_{j}} \times T^{j}_{x_1} \,
\end{equation}
whenever $N\ge 2\pi \lambda^{2j}$; see \cite[eq. 18]{LuR2}.
On the other hand, in \cite[eq. 20]{LuR2} it was proven that for~$k > 2j$, we have \begin{equation}\label{GIPBIS2}
\abs[\Big]{ S_{N} \Big( \frac{t}{2\pi \lambda^{j}} \Big) f_{\theta_{k}}(\bar{x}) }
\lesssim
\lambda^{-k \delta},
\quad
\mbox{for all}
\quad
(\bar{x}, t) \in \R^{n-1} \times T^{j}_{x_1} \, ,
\end{equation}
whenever $N\ge 2\pi \lambda^{2j}$. It is something of a nuisance that this does not quite hold for all $k>j$. To circumvent this, we consider
\begin{equation}\nonumber
X^{k, \delta}_{\lambda^{k-j} t\theta_{k}} :=\lambda^{k(\sigma-1)} \mathbb{Z}^{n-1} + Q(\lambda^{k-j} t \theta_{k},\varepsilon_{2} \lambda^{-k(1 - 2\delta)}) \, .
\end{equation}
In \cite[eq. 19]{LuR2} it was proven that, when $j<k\le 2j$ and $x_1\in(0,1/2)$,
\begin{equation}\label{GIPBIS}
\abs[\Big]{ S_{N} \Big( \frac{t}{2\pi \lambda^{j}} \Big) f_{\theta_{k}}(\bar{x}) }
\lesssim
\lambda^{-k \delta},
\quad
\mbox{for all}
\quad
(\bar{x}, t) \in ( \R^{n-1} \setminus X^{k, \delta}_{\lambda^{k-j} t\theta_{k}} ) \times T^{j}_{x_1} \,
\end{equation}
whenever $N\ge 2\pi \lambda^{2j}$.
Thus, considering
\begin{equation*}
\Gamma_{\!t\theta_{j}}^{j} := X_{t\theta_{j}}^{j} \setminus \bigcup_{j<k \le 2j } X^{k, \delta}_{\lambda^{k-j} t\theta_{k}}\quad \mbox{and}\quad
\Gamma^{j}_{\!x_1} := \bigcup_{ t \in T^{j}_{x_1} } \Gamma_{\!t \theta_{j}}^{j} \, ,
\end{equation*}
an immediate consequence of~\eqref{GalInvPreq}, \eqref{GIPBIS} and~\eqref{GIPBIS2} is that if~$x \in \Gamma^j$, defined by
\begin{equation*}
\Gamma^j=\left\{x\in\R^n \, : \, x_1\in(0,1/2),\quad \bar{x}\in\Gamma^{j}_{\!x_1}\right\} \, ,
\end{equation*}
there exists a time $t_{j}(x) \in T^{j}_{x_1}$ such that
\begin{equation}\nonumber
\mbox{(i)} \
\abs[\Big]{ S_{N} \Big( \frac{t_{j}(x)}{2\pi \lambda^{j}} \Big) f_{\theta_{j}}(\bar{x}) } \gtrsim \lambda^{j \delta };
\qquad
\mbox{(ii)} \
\abs[\Big]{ S_{N} \Big( \frac{t_{j}(x)}{2\pi \lambda^{j}} \Big) f_{\theta_{k}}(\bar{x}) }
\lesssim \lambda^{-k \delta}
\quad
\mbox{for all}
\quad
k > j \, .
\end{equation}


Now divergence occurs on the set of $x$ that belong to infinitely many $\Gamma^{j}$; that is \begin{equation*}
\Gamma := \bigcap_{j\ge1} \bigcup_{k\ge j} \Gamma^{k} \, .
\end{equation*}
To see this, we note that if $x \in \Gamma$ there exists an infinite subset $J(x)\subset\mathbb{N}$ with an associated
sequence of times
$t_{j}(x) \in T^{j}_{x_1}$, for all $j\in J(x)$, such that both (i) and (ii) are satisfied.
The solution factorises as in~\eqref{eq:factrization}, so that, recalling~\eqref{dk},
we see that the properties (i) and (ii) remain true while considering the extension $f_j$, defined by
\[
f_j(x)=e^{i\pi \lambda^jx_1}\phi(\lambda^{j/2}x_1)f_{\theta_{j}}(\bar{x}) \, .
\]
Now, since $u_0=\sum_{j\ge 1} f_j$, by the triangle inequality
\begin{equation}\nonumber
\abs[\Big]{ S_{N} \Big( \frac{t_{j}(x)}{2\pi \lambda^{j}} \Big) u_0(x) }
\gtrsim \abs[\Big]{ S_{N} \Big(\frac{t_{j}(x)}{2\pi \lambda^{j}} \Big) f_j(x)} - \abs{ A_1 } - \abs{ A_2 } \, ,
\end{equation}
where
\begin{equation}\nonumber
A_1 \! := \! \sum_{1 \leq k < j} S_{N} \Big( \frac{t_{j}(x)}{2\pi \lambda^{j}} \Big) f_k(x)
\qquad \text{and}\qquad
A_2 \! := \! \sum_{k > j} S_{N} \Big( \frac{t_{j}(x)}{2\pi \lambda^{j}} \Big) f_k(x) \, .
\end{equation}
We have already proved that, for $x \in \Gamma$,
\[
\abs[\Big]{ S_{N} \Big(\frac{t_{j}(x)}{2\pi \lambda^{j}} \Big) f_j(x) } \gtrsim \lambda^{j \delta }\quad \text{and}\quad
\abs{ A_2 } \leq \sum_{k > j} \lambda^{- k \delta} \ \lesssim \ 1\, .
\]
On the other hand, by bounding the terms trivially and taking $\lambda$ sufficiently large, we can also arrange that
\[
\abs{ A_1} \leq \sum_{1 \leq k < j} \lambda^{k \delta } \leq \frac{1}{2}\abs[\Big]{ S_{N} \Big(\frac{t_{j}(x)}{2\pi \lambda^{j}} \Big) f_j(x)} \, .
\]
Thus, for any $x \in \Gamma $ where the solution is defined, we have
\begin{equation*}
\abs[\Big]{ u \Big(x, \frac{t_{j}(x)}{2\pi \lambda^{j}}\Big) }
= \lim_{N \to \infty}
\abs[\Big]{ S_{N} \Big( \frac{t_{j}(x)}{2\pi \lambda^{j}} \Big) u_0(x) } \gtrsim \lambda^{j \delta } \, ,
\end{equation*}
so
there is a sequence of times $\frac{ t_{j}(x)}{2\pi \lambda^{j}}$
for which
\[
\abs[\Big]{ u \Big( x,\frac{t_{j}(x)}{2\pi \lambda^{j}}\Big) } \to \infty \qquad \mbox{as} \qquad \frac{t_{j}(x)}{2\pi \lambda^{j}} \to 0 \, .
\]

Now, recalling that
$ s < \frac{(n-1) \sigma}{2}+\frac{1}{4} - \delta$, the proof would be complete if we
could prove that the $\alpha$--Hausdorff measure of $\Gamma$
were positive, taking $\delta$ and $\sigma$ sufficiently close to $0$ and $\frac{1+2(n-\alpha) }{2(n+1)}$, respectively. Considering the slices $\Gamma_{\!x_1}$, defined via
\[
\Gamma=\left\{x\in\R^n \, : \, x_1\in(0,1/2),\quad \bar{x}\in\Gamma_{\!x_1}\right\},
\]
it would
suffice to prove that the~$(\alpha-1)$--Hausdorff measure of $\Gamma_{\!x_1}$ is positive for all~$x_1\in(0,1/2)$; see
for instance~\cite[Proposition~7.9]{Falconer2}.
For this we must choose the modulation directions~$\theta_{j} \in \mathbb{S}^{n-2}$ appropriately, via the ergodic argument of the second section ($\theta_{j} \in (0,1)$ if $n=2$).
Note that $X_{t\theta_{j}}^{j}$ is a union of disjoint open cubes of side-length $\varepsilon_{2} \lambda^{-j}$,
while $X^{k, \delta}_{\lambda^{k-j} t\theta_{k}}$ is a union of disjoint closed cubes of side-length $\varepsilon_{2} \lambda^{-(1 - 2\delta)k}$. The
distance between the cubes is approximately $\lambda^{(\sigma-1)j}$ in the case of the former and $\lambda^{(\sigma-1)k}$ in the case of the latter.
Thus we see that
$\Gamma_{\!t\theta_{j}}^{j}$ is a union of disjoint open sets $\calQ(\bar{x}, \varepsilon_{2} \lambda^{-j})$ that we call pseudo-cubes.




\subsubsection*{Case $\alpha=n$} In this case, the $(n-1)$-dimensional Lebesgue measure $\abs{\cdot}$ of the pseudo-cubes is comparable to actual cubes;
\begin{equation}%\label{PBallLebMeas}
\begin{split}\nonumber
\abs{ & \calQ(\bar{x} , \varepsilon_{2} \lambda^{-j}) } \, \geq \, \abs{Q(\bar{x}, \varepsilon_{2} \lambda^{-j})} -
\abs[\Big]{ Q(\bar{x}, \varepsilon_{2} \lambda^{-j}) \cap \!\!\!\! \bigcup_{j<k\le 2j} \!\!\! X^{k, \delta}_{\lambda^{k-j} t\theta_{k}}}
\\
& \simeq \,
\varepsilon_{2}^{n-1} \lambda^{-(n-1)j} - \varepsilon_{2}^{n-1} \lambda^{-(n-1)j} \!\!\!
\sum_{k=j+1}^{2j} \!\!\! \lambda^{-(n-1)(1-2\delta)k} \lambda^{(n-1)(1-\sigma)k}
\gtrsim\ \, \varepsilon_{2}^{n-1} \lambda^{-(n-1)j} \, ,
\end{split}
\end{equation}
where we have taken $\lambda$ sufficiently large (recalling $\delta < \sigma/4$).
Thus, using Corollary~\ref{Corollary:ToroImpr} with $d=n-1$, $\gamma = \alpha-1$ and $R=\lambda^{j}$, we can choose the $\theta_{j}$
so that $\abs{\Gamma^j_{\!x_1}}\gtrsim1$ for all~$x_{1}\in (0,1/2)$, provided that~$j$ is sufficiently large and $\sigma < \frac{n}{2(n+1)}$. From this we see that
\[
\lim_{j\to\infty} \abs[\Big]{\bigcup_{k\ge j}\Gamma^k_{\!x_1}}\gtrsim1 \, ,
\]
and,
since this is a decreasing sequence of sets that are contained in a set with finite $(n-1)$-dimensional Lebesgue measure,
we can conclude that
\[
\abs{\Gamma_{\!x_1}}=\abs[\Big]{\bigcap_{j\ge1}\bigcup_{k\ge j}\Gamma^k_{\!x_1}}\gtrsim 1 \, ,
\]
for all $x_1\in (0,1/2)$.
This completes the proof in the case $\alpha=n$.

\subsubsection*{Case $\alpha<n$}
We will prove that the $\beta$--Hausdorff measure of~$\Gamma_{\!x_1}$ is positive
for any $\beta$ in the
interval $( \frac{(n-1)(2\alpha +1)}{2(n+1)}, \alpha-1 )$.
Note that the interval is not empty if we
restrict to~$\alpha > \frac{3n+1}{4}$. This is enough
to complete the proof, as we could have started with an $\alpha'>\alpha \geq \frac{3n+1}{4}$ that also satisfies
\[
s < \frac{n}{2(n+1)}+\frac{n-1}{2(n+1)}(n-\alpha') \, ,
\]
and performed all of the previous arguments for this $\alpha'$.




Considering the Hausdorff content of a set $E \subset \mathbb{R}^{d}$ defined by
\[
\calH^{\beta}_{\infty}(E) := \inf \Big\{ \sum_{i} \delta_{i}^{\beta} : E \subset \bigcup_{i} Q(x_i, \delta_i)\Big\} \, ,
\]
by the triangle inequality as before, we have that
\begin{equation}\nonumber%\label{PBallFractalMeas}
\begin{split}
\calH^{\beta}_{\infty}(\calQ(\bar{x}, \varepsilon_{2} \lambda^{-j}) ) \, & \geq \, \calH^{\beta}_{\infty} (Q(\bar{x}, \varepsilon_{2} \lambda^{-j})) -
\calH^{\beta}_{\infty} \Big(Q(\bar{x}, \varepsilon_{2} \lambda^{-j}) \cap \!\!\!\! \bigcup_{j<k\le 2k} \!\!\! X^{k, \delta}_{\lambda^{k-j} t\theta_{k}} \Big)
\\
& \gtrsim \,
\varepsilon_{2}^{\beta} \lambda^{-\beta j} \! - \! \varepsilon_{2}^{n-1} \lambda^{-(n-1) j} \!\!\! \sum_{k=j+1}^{2j} \!\!\! \lambda^{- \beta (1-2\delta) k} \lambda^{ (n-1)(1-\sigma)k }
\gtrsim \, \varepsilon_{2}^{\beta} \lambda^{-\beta j} \, ,
\end{split}
\end{equation}
using
$ (1-2\delta) \beta - (n-1)(1 - \sigma) > 0$ and taking~$\lambda$ sufficiently
large. This holds, taking $\delta$ and $\sigma$ close enough to $0$ and~$\frac{1+2(n-\alpha) }{2(n+1)}$,
respectively, since we have restricted to~$\beta > \frac{(n-1)(2\alpha +1)}{2(n+1)}$.
Again we see that, in this range of $\beta$, the $\calH^{\beta}_{\infty}$-content of the pseudo-cubes is comparable to that of the
actual cubes.

We now use Corollary~\ref{Corollary:ToroImpr},
with $d=n-1$, $\gamma = \alpha-1$ and $R=\lambda^{j}$,
to choose~$\theta_j$ such that, for all~$x_{1}\in (0,1/2)$, the
$\Gamma_{\!x_1}^{j}$ are unions of pseudo-cubes whose
centres are~$\varepsilon_{2} \lambda^{-j\frac{\alpha-1}{n-1}}$-dense
in~$B(0,1/2) \subset \R^{n-1}$, when~$j$ is sufficiently large. Recalling that as the sidelengths are shorter, of length $\varepsilon_{2} \lambda^{-j}$, this is not enough to come close to covering the ball as before. However, discarding some pseudo-cubes, if necessary,
we find that~$\Gamma^{j}_{\!x_1}$ contains
a set of pseudo-cubes whose centres are
a {\it quasi-lattice} with separation~$\lambda^{-j\frac{\alpha-1}{n-1}}$; see~\cite[Lemma 4]{LuR2}.
That is to say, for any $\bar{y}\in B(0,1/2) \cap \lambda^{-j\frac{\alpha-1}{n-1}} \mathbb{Z}^{n-1}$
there exists a unique centre~$\bar{x}$
satisfying~$\abs{\bar{x} - \bar{y}} \leq \lambda^{-j\frac{\alpha-1}{n-1}}$.


Using a density theorem due to Falconer~\cite{Falconer1} (see also \cite[Proposition 8.5]{Falconer2} for a similar theorem),
the positivity of the $\beta'$--Hausdorff measure of~$\Gamma_{\!x_1}$, for any $\beta' < \beta$, is a
consequence of the following density property
\begin{equation}\label{WWNTP}
\liminf_{j \to \infty} \calH^{\beta}_{\infty} ( \Gamma^j_{\!x_1} \cap Q(\bar{x}, \delta) ) \geq c\delta^{\beta} ,
\qquad
\forall\ Q(\bar{x}, \delta)\subset B(0,1/2),
\quad
\forall\ \delta >0 \, .
\end{equation}
Thus it would be sufficient for us to show that \eqref{WWNTP} holds for~$\beta \in ( \frac{(n-1)(2\alpha +1)}{2(n+1)}, \alpha-1 )$.
Essentially this means that the most efficient way to cover~$\Gamma^j_{\!x_1} \cap Q(\bar{x}, \delta)$ is with
a single cube of side~$\delta$.
The only real competitor is the cover
that consists of the disjoint union of cubes of side-length $\varepsilon_{2} \lambda^{-j}$ placed on the
top of the pseudo-cubes of the quasi-lattice. However, this cover is costed at
\[
\sum_{i} \delta_i^\beta \simeq \left( \frac{\delta}{ \lambda^{-j\frac{\alpha-1}{n-1}}} \right)^{\!\!n-1} \!\!\! (\varepsilon_{2} \lambda^{-j})^\beta
=\varepsilon_{2}^{\beta}\delta^{n-1} \lambda^{j(\alpha-1-\beta)} \, ,
\]
which diverges as $j\to\infty$ (recalling that $\beta < \alpha-1$).
The remaining coverings are ruled out in exactly the same way as in~\cite[Section 4]{LuR2}.
The only requirement is that the $\calH^{\beta}_{\infty}$-content of the pseudo-cubes is comparable to that
of the actual cubes, which we have already observed, so the proof is complete.
\hfill $\Box$
% Notice that the only requirement of their argument is that
%the $\calH^{\beta}_{\infty}$-outer measure of the pseudo-cubes is comparable to that of the
%actual cubes. Since we have already observed this fact, the proof is complete.
%
%
%\begin{remark}
%The following observations are useful in order to compare the argument here with the one in \cite[Section 4]{LuR2}.
%One should set $d = n-1$ and replace $\alpha$ with $\alpha -1$.
%The set $\Gamma$ considered in \cite{LuR2}, which role here is played by $\Gamma_{\! x_{1}}$,
%is defined slightly differently, with more times in the union. However, only the sidelength and separation
%(given by Corollary~\ref{Corollary:ToroImpr})
%of the sets ({\it pseudo-cubes} in \cite{LuR2}) of the union play a role, so that the argument is exactly the same.
%In order for the $\beta$-Hausdorff measure of the pseudo-cubes to be comparable to that of actual cubes,
%here we
%need that
%$(1-2\delta) \beta - (1 - \sigma) (n-1) > 0$
%is satisfied, provided that we take $\delta$ and $\sigma$ close enough to $0$ and $\frac{1+2(n-\alpha) }{2(n+1)}$, respectively. This is
%the reason of the restriction
%$\beta > \frac{(n-1)(2\alpha +1)}{2(n+1)}$.
%\end{remark}



\begin{thebibliography}{15}



\bibitem{BBCR}
J. A. Barcel\'o, J. Bennett, A. Carbery and K. M. Rogers.
\newblock On the dimension of diverence sets of dispersive equations,
\newblock {Math. Ann.} {\bf349} (2011), 599--622.


\bibitem{BBCRV} J. A. Barcel\'o, J. Bennett, A. Carbery, A. Ruiz and M. C. Vilela, Some special solutions of the Schr\"odinger equation, Indiana Univ. Math. J. {\bf 56} (2007), 1581--1593.


%\bibitem{BR} J. Bennett\ and\ K. M. Rogers, On the size of divergence sets for the Schr\"odinger equation with radial data, Indiana Univ. Math. J. {\bf 61} (2012), no.~1, 1--13.





%\bibitem{Bou3} J. Bourgain, A remark on Schr\"odinger operators, Israel J. Math. {\bf 77} (1992), no.~1-2, 1--16.


%\bibitem{Bou4} J. Bourgain, Some new estimates on oscillatory integrals, in {\it Essays on Fourier analysis in honor of Elias M. Stein (Princeton, NJ, 1991)}, 83--112, Princeton Math. Ser. 42, Princeton.



\bibitem{B} J. Bourgain, On the Schr\"odinger maximal function in higher dimension, Tr. Mat. Inst. Steklova {\bf 280} (2013), 53--66.

\bibitem{Bnew}
J. Bourgain, A note on the Schr\"odinger maximal function, J. Anal. Math. {\bf130} (2016), 393--396.

%\bibitem{Carb} A. Carbery, Radial Fourier multipliers and associated maximal functions, in {\it Recent progress in Fourier analysis (El Escorial, 1983)}, 49--56, North-Holland Math. Stud. 11, North-Holland, Amsterdam.




\bibitem{Carl}
L. Carleson, Some analytic problems related to statistical mechanics,
in {\it Euclidean harmonic analysis (Proc. Sem., Univ. Maryland, College Park, Md., 1979)}, 5--45,
Lecture Notes in Math. {\bf 779}, Springer, Berlin.

%\bibitem{CL} C.-H. Cho\ and\ S. Lee, Dimension of divergence sets for the pointwise convergence of the Schr\"odinger equation, J. Math. Anal. Appl. {\bf 411} (2014), no. 1, 254--260.


%\bibitem{Cow} M. Cowling, Pointwise behavior of solutions to Schr\"odinger equations, in {\it Harmonic analysis (Cortona, 1982)}, 83--90, Lecture Notes in Math. 992, Springer, Berlin.

\bibitem{DahlKenig}
B. E. J. Dahlberg\ and\ C. E. Kenig, A note on the almost everywhere behavior of solutions to the Schr\"odinger equation,
in {\it Harmonic analysis (Minneapolis, Minn., 1981)}, 205--209,
Lecture Notes in Math. {\bf 908}, Springer, Berlin.


\bibitem{DG} C. Demeter\ and\ S. Guo, Schr\"odinger maximal function estimates via the pseudoconformal transformation, (2016), arXiv:1608.07640v1.

\bibitem{DGL} X. Du, L. Guth\ and\ X. Li, A sharp Schr\"odinger maximal estimate in $\R^2$, (2016), arXiv:1612.08946.


\bibitem{Falconer1} K. Falconer, Classes of sets with large intersection, Mathematika {\bf 32} (1985), 191--205.

\bibitem{Falconer2} K. Falconer, Fractal geometry: mathematical foundations and applications, Wiley, 2003.



%\bibitem{LR} S. Lee\ and\ K. M. Rogers, The Schr\"odinger equation along curves and the quantum harmonic oscillator, Adv. Math. {\bf 229} (2012), no.~3, 1359--1379.





\bibitem{GS} G. Gigante\ and\ F. Soria, On the boundedness in $H\sp {1/4}$ of the maximal square function associated with the Schr\"odinger equation, J. Lond. Math. Soc. {\bf 77} (2008), no.~1, 51--68.

\bibitem{L} S. Lee, On pointwise convergence of the solutions to the Schr\"{o}dinger equations in $\mathbb{R}^{2}$, {Int. Math. Res. Not.} (2006), 1--21.



\bibitem{LuR2}
R. Luc\`a\ and\ K. M. Rogers, Coherence on fractals versus convergence for the Schr\"odinger equation, Comm. Math. Phys. {\bf 351} (2017), 341--359.



\bibitem{LuR}
R. Luc\`a\ and\ K. M. Rogers, Average decay for the Fourier transform of measures with applications, J. Eur. Math. Soc., to appear.


\bibitem{M} P. Mattila, Fourier analysis and Hausdorff dimension, Cambridge Studies in Advanced Mathematics 150, Cambridge Univ. Press, Cambridge, 2015.




%\bibitem{MVV1} A. Moyua, A. Vargas\ and\ L. Vega, Schr\"odinger maximal function and restriction properties of the Fourier transform, {Int. Math. Res. Not.} {\bf 16} (1996), 793-- 815.


%\bibitem{MVV2} A. Moyua, A. Vargas\ and\ L. Vega, Restriction theorems and maximal operators related to oscillatory integrals in $\mathbb R\sp 3$, Duke Math. J. {\bf 96} (1999), no.~3, 547--574.



\bibitem{N} E. M. Niki\v sin, A resonance theorem and series in eigenfunctions of the Laplace operator, Izv. Akad. Nauk SSSR Ser. Mat. {\bf 36} (1972), 795--813.



\bibitem{SS} P. Sj\"ogren\ and\ P. Sj\"olin, Convergence properties for the time-dependent Schr\"odinger equation, Ann. Acad. Sci. Fenn. {\bf 14} (1989), no.~1, 13--25.

%\bibitem{Sjolin} P. Sj\"olin, Regularity of solutions to the Schr\"odinger equation, Duke Math. J. {\bf 55} (1987), no.~3, 699--715.

\bibitem{st} E. M. Stein, On limits of sequences of operators, Ann. of Math. {\bf 74} (1961), 140--170.





\bibitem{T} T. Tao, A sharp bilinear restrictions estimate for paraboloids, Geom. Funct. Anal. {\bf 13} (2003), no.~6, 1359--1384.

%\bibitem{TV} T. Tao\ and\ A. Vargas, A bilinear approach to cone multipliers II. Applications, Geom. Funct. Anal. {\bf 10} (2000), no.~1, 185--258.


%\bibitem{Vega} L. Vega, Schr\"odinger equations: pointwise convergence to the initial data, Proc. Amer. Math. Soc. {\bf 102} (1988), no.~4, 874--878.



%\bibitem{Z} {D. \v{Z}ubrini{\'c}}, {Singular sets of Sobolev functions}, {C. R. Math. Acad. Sci. Paris} {\bf 334} (2002), 539--544.

\end{thebibliography}


\end{document}
