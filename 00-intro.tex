\section{Introduction}
In this course we will look at some aspects of Fourier restriction theory that are primarily motivated by applications to partial differential equations.
Here ``Fourier restriction'' will mean that we consider functions whose Fourier support is concentrated near a submanifold.
In order to motivate this topic we will present in this section a context in which such functions arise.

\subsection{Motivation}
Consider the initial value problem for the free Schr\"odinger equation on $\R^{d}$:
\[
2\pi i \partial_{t} \psi = \Delta_{x} \psi,
\quad
\psi(x,0) = g(x).
\]
This is a linear PDE with constant coefficients, so we can represent the solution in terms of the Fourier transform.
Taking the Fourier transform of the equation in the $x$ variable we obtain
\[
2 \pi i \partial_{t} \FT_{x} \psi = (2\pi i \xi)^{2} \FT_{x} \psi(\xi,t).
\]
For each fixed $\xi \in \R^{d}$ this is an ordinary differential equation, and the solution is given by
\[
\FT_{x} \psi(\xi,t) = e(\abs{\xi}^{2} t) \FT_{x} \psi(\xi,0).
\]
Here and later
\[
e(t) := e^{2\pi i t}.
\]
Taking the inverse Fourier transform in the $x$ variable we obtain
\[
\psi(x,t) = \int e(\abs{\xi}^{2}t + \xi x) \hat{g}(\xi) \dif\xi.
\]
The solution $\psi$ can be viewed as a tempered distribution on $\R^{d}\times \R$, and the above formula shows that its Fourier transform (simultaneously in $x$ and $t$) variables is supported on a paraboloid.
Similarly, the solution of the wave equation has Fourier support on a cone.

This provides the motivation for our study of functions with restricted Fourier support.
The abstract results that we will see will eventually lead to a priori estimates for solutions of Schr\"odinger and wave equations.

\subsection{Space localization}
Let $f$ be a function on $\R^{n}$ whose Fourier transform is supported on a submanifold $A$.
Suppose that we want to study the function $f$ locally, say on a ball $B(x,R)$ of radius $R$.
It would be unwise to truncate it to the ball, since this operation would completely destroy any control on the Fourier transform.
One can instead use a smooth truncation.
Let $\phi$ be a Schwartz function such that $\phi \geq \one_{B(0,1)}$ and such that $\widehat{\phi}$ has compact Fourier support.
Then the function $\phi^{x}_{R} := \phi(R^{-1}(\cdot-x))$ is $\geq 1$ on $B(0,R)$, while its Fourier support is contained in a ball of radius $O(R^{-1})$.
It follows that
\[
\widehat{\phi^{x}_{R} f} = \widehat{\phi^{x}_{R}} * \widehat{f}
\]
is supported in a $O(R^{-1})$-neighborhood of the support of $f$.

This space localization trick is one of the main reasons why we will have to consider functions with Fourier support near a submanifold.

\subsection{Frequency localization}
We will study functions with restricted Fourier support by decomposing them into pieces with even smaller Fourier support.
Let us see how this such a function might look like.
Let $f$ be a function on $\R^{2}$ such that $\widehat{f}$ is supported on a piece of parabola of length $\delta \leq 1$ not far from $0$.
\begin{center}
\begin{tikzpicture}[xscale=2,yscale=0.2]
\draw (-1,0) rectangle (1,1);
\draw (-1,1) to[parabola through={(0,0)}] (1,1);
\draw (0,0) node[below] {$\delta$};
\draw (1,0.5) node[right] {$\delta^{2}$};
\end{tikzpicture}
\end{center}
This piece of parabola is contained in a rectangular box of size $\delta \times \delta^{2}$.
Let $\psi$ be a smooth function adapted to this box that identically equals $1$ on the box, so that $\widehat{f} = \widehat{f} \psi$.
Then $f = f * \check{\psi}$, and $\check \psi$ is a smooth function adapted to the polar rectangle of our rectangular box, that is, a rectangle with same orientation and size $\delta^{-1} \times \delta^{-2}$.
Thus, on a heuristical level, the function $f$ is constant at scale $\delta^{-1}$ in the tangential direction of the parabola and constant at scale $\delta^{-2}$ in the orthogonal direction.

In order to most fully make use of this heuristic it makes sense to study the function $f$ at spatial scale $\delta^{-2}$.
This is compatible with the space localization procedure outlined previously: the space localization at scale $\delta^{-2}$ leaves the Fourier support near our box of size $\delta \times \delta^{2}$ (more precisely, the side lengths only have to be enlarged by a constant factor).

So our plan will be to study solutions of the Schr\"odinger equation at scale $R$ by decomposing them into pieces with Fourier support of size $R^{-1/2}$.
