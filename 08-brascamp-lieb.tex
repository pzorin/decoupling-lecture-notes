\section{Brascamp--Lieb inequalities}
We call a finite-dimensional real Hilbert space $H$ with the Lebesgue measure a \emph{Euclidean space}.
For an integer $m \geq 0$ an \emph{$m$-transformation} is a triple
\[
\bfB := (H, (H_j)_{1 \leq j \leq m}, (B_j)_{1 \leq j \leq m})
\]
where $H, H_1,\dotsc,H_m$ are Euclidean spaces and for each $j$, $B_j: H \to H_j$ is a linear transformation.
We only consider $m$-transformations that are \emph{non-degenerate} in the sense that all $B_j$ are surjective, $\dim H_{j} \neq 0$, and $\bigcap_{j=1}^m \ker(B_j) = \Set{0}$.
An \emph{$m$-exponent} is an $m$-tuple $\p = (p_j)_{1 \leq j \leq m} \in (0,\infty)^m$ of non-negative real numbers (one can allow zero entries, but this leads to some case distinctions and is not needed in our application).
A \emph{Brascamp--Lieb (BL) datum} is a pair $(\bfB,\p)$, where $\bfB$ is an
$m$-transformation and $\p$ is an $m$-exponent for some integer $m \geq 0$.
For a BL datum $(\bfB,\p)$ and an $m$-tuple ${\bf{f}} = (f_j)_{1 \leq j \leq m}$ of nonnegative measurable functions $f_j: H_j \to [0,\infty)$ such that $0 < \int_{H_j} f_j < \infty$ we define
\[
\BL(\bfB,\p; \f) :=
\frac{\int_H \prod_{j=1}^m (f_j \circ B_j)^{p_j}}{\prod_{j=1}^m (\int_{H_j} f_j)^{p_j}}.
\]
The \emph{Brascamp--Lieb constant} is the defined as
\[
\BL( \bfB, \p ) :=
\sup_{\f} \BL(\bfB,\p;\f).
\]
Equivalently, $\BL(\bfB,\p)$ is the smallest constant for which the $m$-linear \emph{Brascamp--Lieb inequality}
\begin{equation}\label{BL}
\int_{H} \prod_{j=1}^m (f_j \circ B_j)^{p_j}
\leq \BL( \bfB, \p ) \prod_{j=1}^m (\int_{H_j} f_j)^{p_j}
\end{equation}
holds for nonnegative measurable functions $f_j: H_j \to [0,\infty)$.

\begin{example}
The following inequalities are examples of BL inequalities.
We refer to the introduction of \cite{MR2377493} for a discussion of these examples.
\begin{enumerate}
\item H\"older's inequality: $B_{j}=\id_{H}$ for all $j$,
\item Loomis--Whitney inequality: $H=\R^{m}$, $B_{j}(x_{1},\dotsc,x_{m})=(x_{1},\dotsc,x_{j-1},x_{j},\dotsc,x_{m})$,
\item Young's convolution inequality: $H=\R^{2}$, $m=3$, $B_{1}(x,y)=x$, $B_{2}(x,y)=y$, $B_{3}(x,y)=x+y$.
\end{enumerate}
\end{example}

A central role in the theory of BL inequalities is played by gaussian inputs.
A \emph{gaussian input} is an $m$-tuple $\bfA = (A_{j})_{j=1}^{m}$ of (strictly) positive definite operators $A_{j} : H_{j} \to H_{j}$.
Let $\bff_{\bfA} = (f_{j})_{j}$ with $f_{j}(x) = \exp( -\pi \<A_{j}x,x\> )$ the associated gaussian functions.
Then we can compute
\begin{equation}\label{BLfunctional(not)}
\BLg( \bfB,\p; \A)
:=
\BL( \bfB,\p; \bff_{\A})
=
\left( \frac{\prod_{j=1}^m (\det_{H_j}
A_j)^{p_j}} {\det_H (\sum_{j=1}^m p_j B_j^* A_j B_j)}
\right)^{1/2}.
\end{equation}
We define the \emph{gaussian BL constant} by
\begin{equation}\label{BLfunctional}
\BLg( \bfB,\p) :=
\sup \{ \BLg(\bfB,\p;\A): \A \text{ is a gaussian input for } (\bfB,\p)\}.
\end{equation}
Clearly, $\BLg(\bfB,\p) \leq \BL(\bfB,\p)$.
Surprisingly, in fact equality holds.
\begin{theorem}[{\cite{MR1069246}}]
\label{thm:lieb}
For any BL datum $(\bfB,\p)$, we have
\[
\BL( \bfB, \p ) = \BLg( \bfB, \p ).
\]
\end{theorem}
In the special case of Young's convolution inequality Theorem~\ref{thm:lieb} goes back to \cite{MR0385456} for special $\p$.
In the \emph{rank one} case ($\dim H_{j}=1$ for all $j$) Theorem~\ref{thm:lieb} for general $\p$ was proved in \cite{MR0412366} using rearrangement inequalities in \cite{MR0346109}.
The proof given by Lieb in \cite{MR1069246} has the advantage that it also applies to complex phases, and covers e.g.\ the Fourier transform.
An alternative proof using transportation of mass was given by Barthe \cite{MR1650312}.
Another alternative approach using heat flow was given by Carlen, Lieb, and Loss \cite{MR2077162} in the rank one case and by Bennett, Carbery, Christ, and Tao in \cite{MR2377493}.

We follow the latter approach \cite{MR2377493}.
We will only need and prove Theorem~\ref{thm:lieb} for a special class of BL data called \emph{gaussian extremizable}.

The finiteness of gaussian BL constants is not easy to verify directly.
A necessary and sufficient combinatorial condition was found by Bennett, Carbery, Christ, and Tao.
\begin{theorem}[{\cite{MR2377493}}]
\label{thm:BCCT-g}
Let $(\bfB, \p)$ be a BL datum.
Then $\BLg(\bfB,\p) < \infty$ if and only if we have the scaling condition
\begin{equation}\label{scaling}
\dim(H) = \sum_j p_j \dim(H_j)
\end{equation}
and the dimension condition
\begin{equation}\label{dimension}
\dim(V) \leq \sum_{j=1}^m p_j \dim(B_j V) \text{ for every subspace } V \subseteq H.
\end{equation}
\end{theorem}
We will call \eqref{dimension} the \emph{BCCT condition}.
\begin{remark}
The conditions \eqref{scaling} and \eqref{dimension} imply in particular that $\bfB$ is non-degenerate, as can be seen by testing on $V:=H$ and on $V:=\bigcap_{j=1}^m \ker(B_j)$.
\end{remark}
In the rank one case an equivalent characterization was given in \cite{MR1650312}.

We will only prove Theorem~\ref{thm:BCCT-g} for \emph{simple} data.
\begin{definition}
A BL datum $(\bfB,\p)$ is called \emph{simple} if the scaling condition \eqref{scaling} holds and the BCCT condition \eqref{dimension} holds with strict inequality for all subspaces $V\subset H$ with $0 < \dim V < \dim H$.
\end{definition}

\subsection{Geometric case}\label{geom-sec}

\begin{definition}[Geometric Brascamp--Lieb data] A Brascamp--Lieb datum $(\bfB,\p)$ is said to be \emph{geometric}
if we have
\begin{equation}
\label{eq:BB*=id}
B_j B_j^* = \id_{H_j}
\text{ for every } j \text{ and}
\end{equation}
\begin{equation}\label{pbb}
\sum_{j=1}^m p_j B_j^* B_j = \id_H.
\end{equation}
\end{definition}

\begin{example}
H\"older's and Loomis--Whitney inequalities are geometric BL.
\end{example}

\begin{remark}\label{blg-remark}
Taking traces of the hypothesis \eqref{pbb} we obtain \eqref{scaling}.
One can also verify that \eqref{dimension} holds in this case by multiplying \eqref{pbb} by the orthogonal projection onto $V$ and taking traces.
\end{remark}

For geometric data we will prove the BL inequality using a heat flow.

\begin{proposition}[{\cite{MR1008726,MR1650312}}]
\label{gbl-prop}
Let $(\bfB,\p)$ be a geometric Brascamp--Lieb datum.
Then
\begin{equation}\label{lieb-geom}
\BL(\bfB,\p) = \BLg(\bfB,\p) = 1.
\end{equation}
\end{proposition}

\begin{proof}
Considering the gaussian input $\bfA=(\id_{H_{j}})_{j}$ we see that
\[
1 = \BLg(\bfB,\p;\bfA) \leq \BLg(\bfB,\p).
\]
By~\ref{BL} it therefore suffices to show that for any non-negative measurable functions $f_j: H_j \to [0,\infty)$ we have
\begin{equation}\label{fjpj}
\int_H \prod_{j=1}^m (f_j \circ B_j)^{p_j}
\leq
\prod_{j=1}^m (\int_{H_j} f_j)^{p_j}.
\end{equation}
By Fatou's lemma we may assume that the $f_j$ are smooth and compactly supported.
Now let $u_j: \R^+ \times H \to \R^+$ be the solution to the heat equation
\begin{align}
\label{eq:def-u}
\partial_t u_j(t,x) &= \Delta_H u_j(t,x) \\
\label{eq:initial-u}
u_j(0,x) &= f_j \circ B_j(x)
\end{align}
where $\Delta_H := \div \nabla$ is the usual Laplacian on $H$.
More explicitly, we have
\begin{equation}\label{uj-form}
\begin{split}
u_j(t,x)
&= \frac{1}{(4\pi t)^{\dim(H)/2}} \int_H e^{-\norm{ x-y }_H^2/4t} f_j(B_j y)\ dy
\\ &=
\frac{1}{(4\pi t)^{\dim(H_j)/2}} \int_{H_j} e^{-\norm{ B_j x - z }_{H_j}^2/4t} f_j(z)\ dz.
\end{split}
\end{equation}
Let
\[ Q(t) := \int_{H} \prod_{j=1}^m u_j^{p_j}(t,x)\ dx.\]
It suffices to show the following three inequalities:
\begin{equation}
\label{eq:Q0}
\int_{H} \prod_{j=1}^m (f_j \circ B_j)^{p_j}
\leq
\limsup_{t \to 0^+} Q(t),
\end{equation}
\begin{equation}
\label{eq:Q0<Qinf}
\limsup_{t \to 0^+} Q(t)
\leq
\liminf_{t \to \infty} Q(t),
\quad \text{and}
\end{equation}
\begin{equation}
\label{qinf}
\liminf_{t \to \infty} Q(t)
\leq
\prod_{j=1}^m \left(\int_{H_j} f_j\right)^{p_j}.
\end{equation}
The inequality \eqref{eq:Q0} follows from Fatou's lemma.

Let
\begin{equation}
\label{eq:def-v}
\vec v_j := - \frac{\nabla u_j}{u_j},
\quad
\vec v := \sum_{j=1}^m p_j \vec v_j
\end{equation}
We claim that for every $0<t<\infty$ we have the pointwise inequality
\begin{equation}
\label{eq:1}
\partial_t \bigl( \prod_{j=1}^m u_j^{p_j} \bigr)
\geq
- \div \bigl( \vec v \prod_{j=1}^m u_j^{p_j} \bigr).
\end{equation}
Assuming \eqref{eq:1} we obtain
\[
\partial_{t} Q(t)
\geq
- \int_{H} \div( \vec v \prod_{j=1}^m u_j^{p_j} )
=
0.
\]
The latter equality follows from Gauss's theorem since $\vec v \prod_{j=1}^m u_j^{p_j}$ is rapidly decreasing in space.
Indeed, $\vec v$ grows at most polynomially in space and each $u_{j}$ is rapdily decreasing in the direction of $(\ker B_{j})^{\perp}$.
Since $\bfB$ is non-degenerate, the weighted product of $u_{j}$'s is rapidly decreasing in all directions.
Thus we have reduced \eqref{eq:Q0<Qinf} to \eqref{eq:1}.

Now we show \eqref{eq:1}.
Expanding derivatives on both sides \eqref{eq:1} becomes
\[
\bigl( \prod_{j=1}^m u_j^{p_j} \bigr) \bigl( \sum_{j=1}^{m} \frac{p_{j} \partial_{t} u_{j}}{u_{j}} \bigr)
\geq
- ( \div \vec v ) \bigl( \prod_{j=1}^m u_j^{p_j} \bigr) - \< \vec v , \bigl( \prod_{j=1}^m u_j^{p_j} \bigr) \bigl( \sum_{j=1}^{m} \frac{p_{j} \nabla u_{j}}{u_{j}} \bigr) \>.
\]
Canceling the product of $u_{j}^{p_{j}}$ that is strictly positive for all $t>0$ we see that this is equivalent to
\[
\sum_{j=1}^{m} \frac{p_{j} \partial_{t} u_{j}}{u_{j}}
\geq
- ( \div \vec v ) - \< \vec v , \sum_{j=1}^{m} \frac{p_{j} \nabla u_{j}}{u_{j}} \>.
\]
Inserting the definition \eqref{eq:def-v} we calculate
\[
\div \vec v = - \sum_{j} p_{j} \frac{\Delta u_{j}}{u_{j}} + \sum_{j} p_{j} \frac{\<\nabla u_{j},\nabla u_{j}\>}{u_{j}^{2}}.
\]
Using the equation \eqref{eq:def-u} we see that \eqref{eq:1} is equivalent to
\[
\sum_{j} p_{j} \frac{\<\nabla u_{j},\nabla u_{j}\>}{u_{j}^{2}} + \< \vec v , \sum_{j=1}^{m} \frac{p_{j} \nabla u_{j}}{u_{j}} \>
\geq 0,
\]
which we write in the form
\[
\sum_{j} p_{j} \< \vec{v}_{j} - \vec v, \vec{v}_{j} \> \geq 0.
\]
Since $u_{j}(x)$ depends only on $B_{j}x$ and using \eqref{eq:BB*=id} we see that $\vec{v}_{j} = B_{j}^{*}B_{j} \vec{v}_{j}$, so our claim is equivalent to
\[
\sum_{j} p_{j} \< B_{j}^{*}B_{j} (\vec{v}_{j} - \vec v), \vec{v}_{j} \> \geq 0.
\]
From \eqref{pbb} we have
\[
\sum_{j=1}^m p_j B_j^{*}B_{j} (\vec v - \vec v_j) = \vec v - \sum_{j=1}^m p_j B_j^* B_j \vec v_j = \vec v - \sum_{j=1}^m p_j \vec v_j = 0
\]
and hence the claim is equivalent to
\[
\sum_{j} p_{j} \< B_{j}^{*}B_{j} (\vec{v}_{j} - \vec v), (\vec{v}_{j} - \vec v) \> \geq 0.
\]
This is clearly true, so the claim \eqref{eq:1} holds.
This finishes the proof of \eqref{eq:Q0<Qinf}.

It remains to show \eqref{qinf}.
Using \eqref{uj-form} we can write
\[ Q(t) = \frac{1}{(4\pi t)^{\sum_{j=1}^m p_j \dim(H_j)/2}}
\int_{H} \prod_{j=1}^m \left(\int_{H_j} e^{- \norm{ B_j x - z}_{H_j}^2/4t} f_j(z)\ dz\right)^{p_j}\ dx.\]
Making the change of variables $x = t^{1/2} w$ we obtain
\[ Q(t) = \frac{1}{(4\pi)^{\dim(H)/2}}
\int_{H} \prod_{j=1}^m \left(\int_{H_j} e^{- \norm{ B_j w - t^{-1/2} z}_{H_j}^2/4} f_j(z)\ dz\right)^{p_j}\ dw.\]
Since the $f_j$ are rapidly decreasing and $\bigcap_{j=1}^m \ker(B_j) = \{0\}$, we may then use
dominated convergence to conclude
\begin{align*}
\liminf_{t \to \infty} Q(t) &=
\frac{1}{(4\pi)^{\dim(H)/2}}
\int_H \prod_{j=1}^m \left(\int_{H_j} e^{-\norm{ B_j w }_{H_j}^2/4} f_j(z)\ dz\right)^{p_j}\ dw \\
&=\frac{1}{(4\pi)^{\dim(H)/2}} \prod_{j=1}^m \left(\int_{H_j} f_j\right)^{p_j}
\int_{H} e^{-\< \sum_{j=1}^m p_j B_j^* B_j w, w \>_H/4}\ dw.
\end{align*}
Using \eqref{pbb}, the claim \eqref{qinf} follows.
\end{proof}

\subsection{Gaussian-extremizable case}

\begin{definition}
\label{extremizability}
A Brascamp--Lieb datum $(\bfB,\p)$ is said to be \emph{gaussian-extremizable} if there exists a gaussian input $\A$ for which $\BLg(\bfB,\p) = \BLg(\bfB,\p;\A)$.
\end{definition}

In this section we prove Theorem~\ref{thm:lieb} in the gaussian-extremizable case.

\begin{proposition}
\label{normal-form}
Let $(\bfB,\p)$ be a gaussian extremizable Bras\-camp--Lieb datum.
Then there exist invertible linear transformations $C$ on $H$ and $C_{j}$ on $H_{j}$ such that the BL datum $(\bfB',\p)$ with $\bfB_{j}' = C_{j}B_{j}C$ is geometric.
\end{proposition}

\begin{proof}
Let $\A = (A_j)_{1 \leq j \leq m}$ be a gaussian input for $(\bfB,\p)$ that is a local extremizer for $\BLg(\bfB,\p;\A)$.
Let $M: H \to H$ be given by $M := \sum_{j=1}^m p_j B_j^* A_j B_j$.
Since the $A_j$ are positive definite and $\bfB$ is non-degenerate, the operator $M$ is also positive definite and in particular invertible.
We claim that
\begin{equation}\label{pj-ident}
A_j^{-1} = B_j M^{-1} B_j^* \text{ for all } 1 \leq j \leq m.
\end{equation}
Taking logarithms in \eqref{BLfunctional(not)}, we see that $\A > 0$ is a local maximizer for the quantity
\begin{equation}
\label{eq:log-BLg;A}
\sum_{j=1}^m p_j \log \det_{H_j} A_j - \log \det_H \sum_{j=1}^m p_j B_j^* A_j B_j.
\end{equation}
A Taylor expansion shows that for an invertible matrix $A$ and any matrix $Q$ we have
\[
\frac{\dif}{\dif \epsilon} \log \det (A+\epsilon Q) \abs{_{\epsilon=0}
=
\frac{\dif}{\dif \epsilon} \log \det (1+A^{-1}\epsilon Q) }_{\epsilon=0}
=
\tr (A^{-1} Q).
\]
Differentiating the quantity \eqref{eq:log-BLg;A} in the variable $A_{j}$ in the direction of a self-adjoint transformation $Q_{j} : H_j \to H_j$ we obtain
\[
p_j \tr_{H_j}(A_j^{-1} Q_j) - \tr_H( p_j M^{-1} B_j^* Q_j B_j ) = 0.
\]
We rearrange this using the cyclic property of the trace as
\[
\tr_{H_j}( (A_j^{-1} - B_j M^{-1} B_j^*) Q_j ) = 0.
\]
Since $Q_j$ was an arbitrary self-adjoint transformation, and $A_j^{-1} - B_j M^{-1} B_j^*$ is also self-adjoint, we conclude \eqref{pj-ident}.

Let $B_{j}' := C_{j}B_{j}C$ with $C := M^{-1/2}$ and $C_j := A_j^{-1/2}$.
It remains to show that $(\bfB',\p)$ is a geometric BL datum.
From \eqref{pj-ident} we have $B_j' (B_j')^* = \id_{H_j}$, and from definition of $M$ we have
\[
\sum_{j=1}^m p_j (B_j')^* B_j'
=
\sum_{j=1}^m M^{-1/2} p_j B_j^* A_j B_j M^{-1/2}
=
M^{-1/2} M M^{-1/2}
=
\id_H.
\qedhere
\]
\end{proof}

\begin{corollary}
\label{cor:Lieb:gauss-extr}
Let $(\bfB,\p)$ be a gaussian extremizable Bras\-camp--Lieb datum.
Then
\[
\BL(\bfB,\p) = \BLg(\bfB,\p).
\]
\end{corollary}

\begin{proof}
Both the BL constant and the gaussian BL constant are multiplied by the same Jacobian factor when $B_{j}$ are replaced by $C_{j}B_{j}C$.
Since they coincide for geometric data by Proposition~\ref{gbl-prop}, they also coincide in the setting of Proposition~\ref{normal-form}.
\end{proof}

\subsection{Necessity of BCCT condition}

\begin{lemma}\label{necc}
Let $(\bfB,\p)$ be Brascamp--Lieb datum such that $\BLg(\bfB,\p) < \infty$.
Then the scaling condition \eqref{scaling} and the BCCT condition \eqref{dimension} hold.
\end{lemma}

\begin{proof}
Let $\lambda > 0$ be arbitrary.
Applying \eqref{BLfunctional} with the gaussian input $(\lambda \id_{H_j})_{1 \leq j \leq m}$ we see that
\[\BLg(\bfB,\p)^{2} \geq \lambda^{\sum_{j=1}^m p_j \dim(H_j) - \dim(H)} / \det( \sum_{j=1}^m p_{j} B_j^* B_j ).\]
Since $\lambda$ is arbitrary we obtain \eqref{scaling}.

Next, let $V$ be any subspace in $H$, and let $\lambda>0$.
Let $\A^{(\lambda)} = (A_j)_{1 \leq j \leq m}$ be the gaussian input $A_j := \lambda \id_{B_j V} + \id_{(B_j V)^\perp}$.
Then $\det_{H_j}(A_j) = \lambda^{\dim(B_j V)}$.
Also, we see that $\sum_{j=1}^m p_{j} B_j^* A_j B_j$ is bounded uniformly in $\lambda \leq 1$, and when restricted to $V$ decays linearly in $\lambda$.
Thus $\det_H(\sum_{j=1}^m B_j^* A_j B_j) \lesssim \lambda^{\dim(V)}$.
Inserting this in \eqref{BLfunctional(not)} we obtain
\[
\infty
>
\BLg(\bfB,\p)^{2}
\geq
\BLg(\bfB,\p; \A^{(\lambda)})^{2}
\gtrsim
\frac{\prod_{j=1}^{m} \lambda^{p_{j} \dim(B_{j} V)}}{\lambda^{\dim V}}.
\]
Since this holds for all $0 < \lambda \leq 1$ we obtain \eqref{dimension}.
\end{proof}


\subsection{Sufficiency of BCCT condition for simple data}
\label{sec:gaussian-BL-finite}

In this section we prove Theorem~\ref{thm:BCCT-g} for simple data, and in fact a more precise statement.

\begin{lemma}\label{babygauss} Let $(\bfB,\p)$ be a non-degenerate Brascamp--Lieb datum such that \eqref{dimension} holds.
Then there exists a real number $c > 0$, such that for every orthonormal basis $e_1,\ldots,e_n$ of $H$ there exist sets $I_j \subseteq \{1,\ldots,n\}$ for each $1 \leq j \leq m$ with $\abs{I_j} = \dim(H_j)$ such that
\begin{equation}\label{p-comb-alt}
\sum_{j=1}^m p_j \abs{I_j \cap \{k+1,\ldots,n\}} \geq n-k \text{ for all } 0 < k < n.
\end{equation}
and
\begin{equation}\label{bjei}
\norm{ \bigwedge_{i \in I_j} B_j e_i }_{H_{j}} \geq c \text{ for all } 1 \leq j \leq m.
\end{equation}
If furthermore $(\bfB,\p)$ is simple, then we can choose $I_{j}$ so that the inequality \eqref{p-comb-alt} is strict.
\end{lemma}

\begin{proof}
We claim that it suffices to show a weaker statement, namely that for every orthonormal basis $(\tilde{e}_{i})$ there exist $I_{j}$ as above with
the conclusion by the weaker statement
\[
\bigwedge_{i \in I_j} B_j \tilde{e}_i \neq 0 \text{ for all } 1 \leq j \leq m.
\]
Then by continuity we have \eqref{bjei} for $(e_{i})$ in a small neighborhood of $(\tilde{e}_{i})$ with the same sets $I_{j}$.
Since the space of all orthonormal bases is compact this implies the conclusion.

Hence it suffices to make the vectors $(B_j e_i)_{i \in I_j}$ linearly independent for each $j$.
We select the $I_j$ by a backwards greedy algorithm:
\[
I_{j} :=
\Set[\big]{ i \given B_{j}e_{i} \not\in \lin \Set{ B_j e_{i'} \given i < i' \leq n } }.
\]
Then by backward induction on $i$ we see that $\Set{ B_j e_{i'} \given i \leq i' \leq n, i' \in I_{j} }$ is a basis of the subspace $\lin \Set{ B_j e_{i'} \given i \leq i' \leq n }$.
Since the $B_j$ are surjective it follows that $\abs{I_j} = \dim(H_j)$.
To prove \eqref{p-comb-alt}, we apply the hypothesis
\eqref{dimension} with $V = \lin \Set{ e_{k+1},\dotsc,e_n }$, to obtain
\begin{align*}
n-k = \dim(V)
&\leq
\sum_j p_j \dim(B_j V)
\\ &=
\sum_j p_j \abs{I_j \cap \Set{k+1,\dotsc,n}}.
\end{align*}
If the datum is simple, then the above inequality is strict for $0 < k < n$.
\end{proof}

\begin{proposition}\label{sufbl-meat}
Let $(\tilde{\bfB},\p)$ be a Brascamp--Lieb datum such that \eqref{scaling} and \eqref{dimension} hold.
Then there exists a neighborhood $\calB$ of $\tilde{\bfB}$ such that $\sup_{\bfB \in \calB} \BLg(\bfB,\p) < \infty$.
Furthermore, if $(\tilde{\bfB},\p)$ is simple, then $(\bfB,\p)$ is gaussian-extremizable for every $\bfB \in \calB$.
\end{proposition}

Proposition~\ref{sufbl-meat} combined with Corollary~\ref{cor:Lieb:gauss-extr} in particular shows the sufficiency of \eqref{scaling} and \eqref{dimension} in Theorem~\ref{thm:BCCT-g} in the case of simple data.
In fact it provides the more precise conclusion that the gaussian BL constant is uniformly bounded on a neighborhood of the original $m$-transformation.
This was first made explicit in \cite{MR3783217}.
More precise result on regularity of $\BL(\bfB,\p)$ as a function of $\bfB$ appear in \cite{MR2836590,MR3723636,arxiv:1811.11052}.

\begin{proof}
The left-hand side of \eqref{bjei} is a Lipschitz function of $\bfB$, so applying Lemma~\ref{babygauss} to the datum $(\tilde{\bfB},\p)$ and replacing $c$ by a slightly smaller number $c(\tilde{\bfB})$ we may assume that \eqref{bjei} continues to hold for all $\bfB$ in a small neighborhood $\calB$ of $\tilde{\bfB}$.

Let $\A = (A_j)_{1 \leq j \leq m}$ be a gaussian input and $M := \sum_j p_j B_j^* A_j B_j$.
We want to estimate
\[
\BLg( \bfB,\p; \A)^{2} = \frac{ \prod_{j=1}^{m} (\det A_{j})^{p_{j}}}{\det M}.
\]
The transformation $M$ is self-adjoint; since $\bfB$ is non-degenerate and $p_j > 0$, we also see that it is positive definite.
Thus by choosing an appropriate orthonormal basis $\{e_1,\ldots,e_n\} \subset H$ we may assume that $M = \diag(\lambda_1, \ldots, \lambda_n)$ for some $\lambda_1 \geq \ldots \geq \lambda_n > 0$.

Applying Lemma~\ref{babygauss}, we can find $I_j \subseteq \{1,\ldots,n\}$ for each $1 \leq j \leq m$
of cardinality $\abs{I_j} = \dim(H_j)$
obeying \eqref{p-comb-alt} and \eqref{bjei}.
For each $i \in I_j$, we have
\[ \< A_j B_j e_i, B_j e_i \>_{H_j} = \< e_i, B_j^* A_j B_j e_i \>_{H}
\leq \frac{1}{p_j} \< e_i, Me_i \>_{H} = \lambda_i / p_j.\]
Since $A_{j}$ is positive definite and by the Cauchy--Schwarz inequality this implies
\[
\abs{\< A_j B_j e_i, B_j e_{i'} \>_{H_j}} \leq (\lambda_{i} \lambda_{i'})^{1/2}/p_{j}.
\]
It follows that
\[
\abs{ \det (\< A_j B_j e_i, B_j e_{i'} \>_{H_j})_{i,i' \in I_{j}} }
\leq
\sum_{\sigma \in S_{I_{j}}} \prod_{i\in I_{j}}(\lambda_{i} \lambda_{\sigma(i)})^{1/2}/p_{j}
=
(n_{j}!) p_{j}^{-n_{j}} \prod_{i} \lambda_{i}.
\]
On the other hand, from \eqref{bjei} we see that $(B_j e_i)_{i \in I_j}$ is a basis of $H_j$ with a lower bound on the degeneracy.
We thus conclude that
\[
\det(A_j)
\leq \norm{ \bigwedge_{i \in I_j} B_j e_i }_H^{-2} \abs{ \det (\< A_j B_j e_i, B_j e_{i'} \>_{H_j})_{i,j} }
\leq
C_{\bfn,\bfp} c(\tilde{\bfB})^{-2} \prod_{i \in I_j} \lambda_i.
\]
Thus
\[
\prod_{j=1}^m (\det A_j)^{p_j}
\lesssim
\prod_{i=1}^n \lambda_i^{\sum_{j=1}^m p_j \abs{I_j \cap \{i\}}}.
\]
with an implicit constant depending only on $\bfn,\p,\tilde{\bfB}$, but not on $\bfB$.
We can telescope the right-hand side (using \eqref{scaling}) and obtain
\[ \prod_{j=1}^m (\det A_j)^{p_j}
\lesssim
\lambda_{1}^{n} \prod_{1 \leq k \leq {n-1}} (\lambda_{k+1}/\lambda_k)^{\sum_{j=1}^m p_j \abs{I_j \cap \{k+1,\ldots,n\}}}.\]
Applying \eqref{p-comb-alt} we conclude
\[ \prod_{j=1}^m (\det A_j)^{p_j}
\lesssim
\lambda_{1}^{n} \prod_{1 \leq k \leq {n-1}} (\lambda_{k+1}/\lambda_k)^{n-k}\]
which by reversing the telescoping becomes
\[ \prod_{j=1}^m (\det A_j)^{p_j}
\lesssim
\lambda_1 \ldots \lambda_k
=
\det(M),\]
so that $\BLg(\bfB,\p; \bfA)^{2} \leq C(\bfn,\p,\tilde{\bfB})$.
Since $\bfA$ was arbitrary, by definition \eqref{BLfunctional(not)} we conclude that $\BLg(\bfB,\p)^{2} \leq C(\bfn,\p,\tilde{\bfB})$.

Now suppose that $(\tilde{\bfB},\p)$ is simple.
Then we have strict inequality in \eqref{p-comb-alt}.
Then it follows that
\[
\sum_{j=1}^m p_j \abs{I_j \cap \{k+1,\ldots,n\}} \geq n-k+c' \text{ for all } 0 < k < n
\]
with some strictly positive $c'>0$ independent of $\bfB$ since the cardinalities on the left-hand side can only assume finitely many values.
We may thus refine the above analysis and conclude that
\[ \prod_{j=1}^m (\det A_j)^{p_j}
\leq C(\tilde{\bfB})
\det(M) \prod_{1 \leq k \leq {n-1}} (\lambda_{k+1}/\lambda_k)^{c'}.\]
This shows that $\BLg( \bfB,\p; \A) \to 0$ whenever $\lambda_n / \lambda_1 \to 0$.
Thus to evaluate the supremum it suffices to do so in the region $\lambda_1 \leq C \lambda_n$.
By the scaling hypothesis \eqref{scaling} we may normalize $\lambda_n = 1$.
This means that $M$ is now bounded above and below, which by surjectivity of $B_j$ implies that $A_j$ is also bounded.
We may now also assume that $A_j$ is bounded from below since otherwise $\BLg( \bfB,\p; \A)$ will be small.
We have thus localized each the $A_j$ to a compact set, and hence by continuity we see that an extremizer of $\A \mapsto \BLg( \bfB,\p; \A)$ exists.
Thus $(\bfB,\p)$ is gaussian-extremizable as desired.
\end{proof}

\begin{corollary}[Locally uniform BL inequality]
Let $(\tilde{\bfB},\p)$ be a \emph{simple} Brascamp--Lieb datum such that \eqref{scaling} and \eqref{dimension} hold.
Then there exists a neighborhood $\calB$ of $\tilde{\bfB}$ such that
\[
\sup_{\bfB \in \calB} \BL(\bfB,\p) < \infty.
\]
\end{corollary}
\begin{proof}
By Proposition~\ref{babygauss} we know that the gaussian BL constant is bounded uniformly on a neighborhood $\calB$ and each datum $(\bfB,\p)$ with $\bfB\in\calB$ is gaussian extremizable.
By Corollary~\ref{cor:Lieb:gauss-extr} the gaussian BL constants coincide with the full BL constants on this neighborhood.
\end{proof}

The argument in \cite{MR2377493} for non-simple data is based on a decomposition of non-simple data as a kind of semidirect product of simple data.
We will be content with the simple case.
